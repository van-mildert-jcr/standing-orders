% Set font size to 12; article is default LaTeX doc type and meets our purposes
\documentclass[12pt]{article}
\usepackage{mildert}

\title{Operational Standing Orders}
\author{Van~Mildert~College Junior~Common~Room}
\newcommand{\thedate}{7th October 2022}
\date{\thedate}

\begin{document}
\begin{titlepage}  % Title page
    \maketitle
    \begin{figure}[h]
    \includegraphics[scale=0.25]{arms}  % Coat of arms
    \centering
    \end{figure}
    \thispagestyle{empty}
\end{titlepage}

\setcounter{page}{2}  % Correct page numbering
\section*{Introduction}
These are the Operational Standing Orders of Van Mildert College Junior Common Room, approved on \thedate.

This version was written by the Steering Committee of 2017-18:\\
\hspace*{2cm}Georgina Robinson (JCR Chair) (HLM)\\
\hspace*{2cm}Alastair Hargreaves-McManus\\
\hspace*{2cm}George Singleton\\
\hspace*{2cm}James Smith\\
\hspace*{2cm}Amelia Spencer (JCR Vice-Chair)

Amended in October 2018 by the Executive Committee, Sheehan Quirke (JCR President) and Alastair Hargreaves-McManus.\\
Amended in June 2019 by Alastair Hargreaves-McManus and Sheehan Quirke (JCR President).\\
Amended in June 2020 by the Executive Committee.\\
Amended in November 2021 by Dominic Laurens (JCR Chair).\\
Amended in February 2022 by the Executive Committee and Constitutional Committee.\\
Amended in May 2022 by the Executive Committee and Constitutional Committee.\\
Amended in October 2022 by Dominic Laurens (JCR Chair), and the Executive and Constitutional Committees.

\newpage
\tableofcontents{}
\newpage


\section{Meetings}
\begin{enumerate}
    \subsection{The Chair}
    \item Where this section (`Meetings') refers to the ``Chair'' and not the ``JCR Chair'' explicitly, then it refers to the member chairing the meeting as necessary.
    \item The JCR Chair must chair all meetings unless they are absent or feel that it is inappropriate.
    \item The Chair must step down during an item of business in which they have a personal interest.
    \begin{enumerate}
        \item This clause may be waived if it would cause unreasonable disruption.
    \end{enumerate}
    \item If the JCR Chair is unable to chair a meeting or item of business, then one of the following officers must chair it (in order of precedence):
    \begin{enumerate}[(a)]
        \item The Deputy Chair
        \item The President
        \item Any other member of the Executive Committee
    \end{enumerate}
    \subsection{Conduct and Discipline}
    \item The Chair must act impartially and in accordance with the Standing Orders.
    \item The Chair takes precedence over all other speakers.
    \item The Chair chooses who may speak at any given time.
    \begin{enumerate}
        \item The Chair must, where reasonable, give preference to those who have not spoken already on the matter under debate.
    \end{enumerate}
    \item The Chair may expel any member who is causing unreasonable disruption to the meeting.
    \item All members must behave in a respectful and courteous manner during meetings.
    \item All speeches must be relevant to the current business.
    \subsection{Attendance}
    \item All members of the JCR have the right to attend and speak at JCR meetings.
    \item Any person who is not a member of the JCR may attend and speak at a JCR meeting with permission of the Chair when it does not influence JCR business; for example, when candidates for roles in the Students' Union canvass at a JCR meeting.
    \subsection{The Agenda}
    \item An agenda must be prepared for the meeting describing the planned order of business.
    \begin{enumerate}
        \item The content of the agenda is decided by the Chair.
        \item All matters to be included on the agenda for a meeting must be sent to the Chair.
        \item The agenda must conclude with ``Any Other Business''.
    \end{enumerate}
    \subsection{Notice}
    \item At least five days' notice must be given for an ordinary meeting of the JCR.
    \item At least three days' notice must be given for an extraordinary meeting of the JCR, which may be called by:
    \begin{enumerate}[(a)]
        \item A signed request from JCR members with the signatures of at least 10\% of JCR members.
        \item JCR Executive Committee.
    \end{enumerate}
    \item At least 24 hours' notice must be given for an emergency meeting, which may be called by the Executive Committee in the case where an immediate decision by the JCR is required.
    \item Where ``notice'' is required to be given for the purposes of clauses 1.14-1.16, this must be given by an email to all JCR members and by a post on the JCR Facebook group. It is the joint responsibility of the Chair and Communications Officer that such notice is given in sufficient time.
    \item Where ``notice'' is required to be given for the purposes of clauses 1.14-1.16, this must include the agenda for the meeting. If any changes are subsequently made to the agenda prior to the meeting, the Chair should make this known at an appropriate time.
    \subsection{Motions}
    \item All motions must be supported by two members of the JCR or by a recognised JCR committee, club or society.
    \item A motion that does not appear on the agenda may be brought under ``Any Other Business'' and discussed at the discretion of the Chair, except for the following types of motion:
    \begin{enumerate}[(a)]
        \item A motion to authorise the use of JCR funds.
        \item A motion to amend the Standing Orders.
        \item A motion of no confidence.
        \item A motion to give a formal warning.
    \end{enumerate}
    \item A procedural motion, as defined by the relevant clauses below, may be proposed at any time.
    \item Any member of the JCR may propose an amendment to a motion. If this is seconded by another member the meeting will debate the amendment and put it to a vote.
    \begin{enumerate}
        \item An amendment may not be amended. The original amendment must be put to a vote before a second amendment can be discussed.
    \end{enumerate}
    \item The Chair may amend motions without vote to correct a printing or grammatical error only.
    \item A motion of no confidence can only be moved by:
    \begin{enumerate}[(a)]
        \item The President,
        \item Governance Committee, \emph{or}
        \item A member with the supporting signatures of at least 10\% of JCR members.
    \end{enumerate}
    \item A motion of no confidence or formal warning must be approved by a JCR meeting and then by a referendum.
    \begin{enumerate}
        \item In the referendum, the motion must receive at least two-thirds of the vote to be approved.
    \end{enumerate}
    \subsection{General Procedure}
    \item Only one matter at a time may be discussed.
    \item Any member of the JCR may raise a point of order which must take precedence over all other business except when a speech is being delivered.
    \begin{enumerate}
        \item A point of order raised during a debate must relate to the conduct of the meeting at that time.
    \end{enumerate}
    \item Any member of the JCR may raise a point of information to offer strictly factual information relating to the current business.
    \begin{enumerate}
        \item A speaker may choose to accept a point of information immediately or direct it to be heard at the end of their speech.
        \item The Chair may refuse to allow further points of information if they are being used inappropriately.
    \end{enumerate}
    \item For all motions unless the Standing Orders specify otherwise, the procedure is as described in this clause:
    \begin{enumerate}
        \item The proposer (or their representative) must give a speech explaining the motion. Members may then ask questions about the motion.
        \item If a member offers formal opposition then they may give a speech against the motion. If multiple members offer formal opposition, then the Chair shall choose the member that appears to have the strongest support from those affected by the motion in question.
        \item The motion will then be debated by the meeting. This debate will last as long as the Chair feels is reasonable.
        \item During the debate, amendments to the motion may be proposed and debated in accordance with the relevant clauses above.
        \item Following the debate, the proposer (or their representative) and the formal opposer may give speeches of summation.
        \item Finally, the motion shall be put to a vote of the meeting or put to a referendum if required by the Standing Orders or at the discretion of the Chair.
    \end{enumerate}
    \subsection{Procedural Motions}
    \item A procedural motion is defined as one that falls into one of the following categories:
    \begin{enumerate}[(a)]
        \item A motion to challenge a ruling of the Chair
        \item A motion to remove an item from the agenda
        \item A motion to hold an item until the next meeting
        \item A motion to end the current debate
        \item A motion to take the motion under debate in parts
        \item A motion to order that the current motion be put to a referendum
        \item Any other motion relating to the conduct of the meeting, at the discretion of the Chair
    \end{enumerate}
    \item A procedural motion cannot go to a referendum and must be discussed within the meeting itself. For the motion to pass, it must meet a quota of 50\% of the votes.
    \item Any member of the JCR may propose a procedural motion which must be seconded by another member. The motion must be relevant to the current business.
    \item A procedural motion takes precedence over any other business. If more than one procedural motion is moved, then their order of precedence is that of the list of categories in the clause defining procedural motions above.
    \item A procedural motion is binding unless it contradicts the Standing Orders.
    \item If a procedural motion is moved to challenge a ruling of the Chair, then the procedure is as follows:
    \begin{enumerate}
        \item The Chair will be deemed automatically to have a personal interest in the debate and must act accordingly.
        \item The proposer must give a short speech setting out their challenge and the (challenged) Chair may give a speech setting out their defence.
        \item The motion will then be voted on by the meeting or withdrawn at the choice of the proposer.
        \item The motion will be approved only if it receives a two-thirds majority of the vote.
        \item If the motion is approved, then the challenged ruling is invalid and the (challenged) Chair must vacate the Chair for the rest of the item of business relating to the challenged ruling.
    \end{enumerate}
    \subsection{Hustings}
    \item The procedure for hustings is as described in this clause:
    \begin{enumerate}
        \item The order of candidates in the hustings shall be decided randomly by the Chair.
        \item Each candidate's proposer shall give a one-minute speech followed by the candidate giving a two-minute speech.
        \item Each candidate must sing a song, perform a poem, or tell a joke. This may be done with the proposer and/or seconder of the candidate. This may be done before or after their speeches; order to be discussed with the Chair prior to the meeting.
        \item In the case of Presidential hustings, each candidate's campaign video (if approved) will then be played to the meeting.
        \item Each candidate may then each ask one question.
        \item The incumbent may then ask questions. The incumbent must act impartially.
        \item Questions will then be taken from the floor. Members of the Executive Committee may ask questions before other members.
        \item Candidates may then ask further questions at the discretion of the Chair.
        \item In the case of Presidential hustings, each candidate may finally make some concise final comments.
        \item Finally, the statement for RON (if one exists) is read by the Chair or the permitted member.
    \end{enumerate}
    \item Questions may be addressed at individual candidates or at all candidates.
    \item Length of answers permitted are to be determined at the discretion of the Chair.
    \item In the event of a question addressed at an individual candidate, the other candidates may also make a comment in response.
    \item Any hustings must have been on the agenda at the start of the meeting to be valid.
    \subsection{Minutes}
    \item The minutes of JCR Meetings must (as a minimum) include the following details:
    \begin{enumerate}[(a)]
        \item The date and time of the meeting;
        \item The Chair of the meeting;
        \item All votes and decisions of the meeting;
        \item If the meeting was quorate.
    \end{enumerate}
\end{enumerate}
\newpage

\section{Elections and Referenda}
\begin{enumerate}
    \item For the purposes of this section (`Elections and Referenda'), where ``notice'' is required to be given, this must be given by an email to all JCR members and by a post on the JCR Facebook group.
    \item Adequate notice for elections and referenda must be given.
    \begin{enumerate}
        \item For elections, notice must be given at least 7 days before the opening of nominations.
        \item For referenda, notice must be given at least 3 days before the opening of voting.
    \end{enumerate}
    \item For every ballot, the roles of Senior and Junior Returning Officers must be filled by two of the following Officers (in order of precedence):
    \begin{enumerate}[(a)]
        \item The Chair
        \item The President
        \item Any other member of the Executive Committee.
    \end{enumerate}
    \item For every ballot, the role of Scrutineer must not be held by an Executive Officer and must be filled by one of the following (in order of precedence):
    \begin{enumerate}[(a)]
        \item The Deputy Chair (as Assistant Head of Governance Committee)
        \item A member of Governance Committee
        \item A member of the JCR who is deemed trustworthy by the President.
    \end{enumerate}
    \item The President must determine who will fill the roles of the Returning Officers and Scrutineer more than 24 hours before the opening of nominations.
    \begin{enumerate}
        \item A member chosen to fill one of these roles must immediately confirm to the President that they are able to do so.
        \item By accepting a role, a member forfeits their right to run in the election.
        \item The Chair and Deputy Chair should disclose to the President in advance if they intend to run for election.
    \end{enumerate}
    \item The names of the Returning Officers and Scrutineer must be included in the notice for the ballot and, for elections, in the announcement of the opening of nominations.
    \begin{enumerate}
        \item The following text must accompany the names:\\
        ``\textrm{The Scrutineer will make sure the election is conducted fairly and enforce the constitution. If you have any complaints or worries about the election, please refer them to the Scrutineer, who will deal with the issue and take it to the Governance Committee for consideration.}''
    \end{enumerate}
    \item The voting for all ballots must be conducted using the online voting system provided by the University.
    \begin{enumerate}
        \item If a member is unable to vote using this method, then they may make their vote by signed letter or email from their official University account sent to the Senior Returning Officer.
        \item An Honorary Life Member of the JCR may vote via signed letter or an email sent to the Senior Returning Officer.
        \item These methods of voting must be advertised at the election hustings.
    \end{enumerate}
    \item Voting for a ballot will be open for a period set by the Senior Returning Officer which must be no shorter than 24 hours and no longer than 168 hours (7 days).
    \begin{enumerate}
        \item Notice of the times of the start and end of the period must be given at least 48 hours before the start of voting.
        \item Notice must be also given at the start of the voting period.
    \end{enumerate}
    \item The Returning Officers must announce the results of a ballot on the stairs leading down to the JCR and must then publish those results.
    \begin{enumerate}
        \item The Returning Officers must also publish the results on the online voting system.
    \end{enumerate}
    \subsection{Referenda}
    \item A referendum must be held within three weeks of the relevant JCR meeting.
    \begin{enumerate}
        \item The Chair is responsible for ensuring that this clause is adhered to.
    \end{enumerate}
    \item For all referenda, a copy of the motion together with all other relevant information must be included when notice is given and must be available on the online voting system.
    \subsection{Elections}
    \item Governance Committee must decide an election timetable at the start of each term.
    \begin{enumerate}
        \item The timetables must ensure that elections happen sufficiently in advance of the starts of the terms of office.
    \end{enumerate}
    \item Nominations for elections must be submitted by email to the Senior Returning Officer in the time period specified in the election timetable.
    \begin{enumerate}
        \item The time period specified must be no shorter than 7 days unless Governance Committee concludes that there are extraordinary circumstances that make a shorter period necessary for the good governance of the JCR.
        \item The Senior Returning Officer must announce the opening of nominations.
        \item The announcement must include the time of the close of nominations, how to submit a nomination and details of the position(s) up for election.
        \item Nominations must contain the names of a proposer and a seconder who are members of the JCR and not members of the Executive Committee.
        \item Nominations must be accompanied by a manifesto not exceeding one side of A4 paper. Candidates may also submit a plain text version of their manifesto for accessibility.
        \item Nominations for a sabbatical role must also be accompanied by a policy statement not exceeding one side of A4 paper.
        \item If an invalid nomination is submitted, the Returning Officers must inform the nominee and give a reason as soon as possible and allow them to submit a new nomination before the candidates are announced.
    \end{enumerate}
    \item The Returning Officers must announce the candidates between 12 and 36 hours after the close of nominations.
    \item Candidates may also submit a 200 word statement for display on the online voting system to the Senior Returning Officer before the start of voting.
    \item All candidates standing for election to be President, FACSO, Senior Welfare Officer, or Senior Freshers' Representative must meet with the College Principal (or their delegate) before submitting their nomination.
    \begin{enumerate}
        \item The College Principal (or their delegate) must raise any concerns about a candidate with the Returning Officers of the election and the External Trustees.
        \item The JCR may take appropriate action, depending on the severity of the concerns raised.
    \end{enumerate}
    \item Candidates must take part in hustings or are be deemed to have withdrawn from the election.
    \begin{enumerate}
        \item Candidates may take part in hustings remotely by video call, or similar, if their physical attendance is not possible.
        \item The Senior Returning Officer may waive this clause at their discretion in exceptional circumstances.
    \end{enumerate}
    \item Accurate minutes of the hustings must be taken and published on the JCR Facebook group before the start of voting.
    \item All elections will be conducted by the Single Transferable Vote system in accordance with the rules set out in Appendix J.
    \item In an uncontested election, RON is considered to have won if it receives over 33\% of the vote.
    \item In a contested election, RON will be treated as a candidate for the purposes of voting procedures.
    \item The election process will be restarted at a reasonable future date set by the Chair in consultation with Governance Committee and the Executive Committee, in the event of any of the following occurrences:
    \begin{enumerate}[(a)]
        \item No valid nominations are received
        \item The election is inquorate
        \item RON wins the election
    \end{enumerate}
    \subsection{Campaigning and Publicity}
    \item In elections for Executive Officers, candidates may campaign in-person during the period after nominations are announced and before the opening of voting takes place.
    \begin{enumerate}
        \item In-person campaigning must be monitored by the Scrutineer of that election.
        \item\label{canvassing-limit} The maximum total time spent campaigning in-person is 4.5 hours.
        \item Any given single session of in-person campaigning will be a minimum of 0.5 hours for the purpose of \ref{canvassing-limit}.
        \item Time restrictions may be waived by Governance Committee if fair and beneficial to democracy.
        \item In-person campaigning must only take place in communal areas of College that any JCR member can access.
        \item The proposer and seconder of the candidate may accompany the candidate while campaigning in-person.
    \end{enumerate}
    \item In elections for Executive Officers, candidates may campaign online, during the period after nominations have been announced.
    \begin{enumerate}
        \item Only the candidate, their proposer, and their seconder may campaign online.
        \item All online campaigning must adhere to the requirements set ou in Appendix C.
    \end{enumerate}
    \item Candidates, within reason, must not spend any money on campaigning.
    \item All approved manifestos will be printed by the Senior Returning Officer.
    \begin{enumerate}
        \item The same number of manifestos must be printed for each candidate.
        \item Candidates must only place their manifestos in the communal toilets of the JCR and residential blocks, excluding Deerness.
    \end{enumerate}
    \item All approved manifestos will be posted on the JCR Facebook group by the Senior Returning Officer after the closing of nominations.
    \item Candidates husting for the role of President must produce a three-minute campaign video to be approved by Governance Committee before it is shown at the hustings.
    \begin{enumerate}
        \item The video must be sent to the Head of Governance Committee at least 48 hours before the hustings.
        \item Governance Committee must give prompt feedback so there is time for any necessary changes to be made.
        \item Once shown at the hustings, campaign videos may be published online.
    \end{enumerate}
    \item In an election, a member may only campaign on behalf of RON with permission of the Senior Returning Officer.
    \begin{enumerate}
        \item The Senior Returning Officer should give permission to the first member who requests it.
        \item No more than one member can be given permission.
        \item A request to campaign on behalf of RON must be made before the day voting starts.
        \item Individuals conducting a campaign on behalf of RON have the right to complete anonymity, if requested.
        \item The individual representing RON must submit a statement, not exceeding one side of A4, to be read by the Senior Returning Officer at the end of the hustings. The individual must not take part in the hustings.
        \item The individual representing RON must not canvass.
        \item Unless otherwise stated in the Standing Orders, a member campaigning for RON is subject to the same election regulations as any other candidate. This includes campaign material entitlements, with the hustings statement acting as a manifesto.
    \end{enumerate}
    \item Campaigning on behalf of candidates or potential candidates for election is forbidden.
\end{enumerate}
\newpage

\section{Appointment by Interview}
\begin{enumerate}
    \item The process of appointment by interview must be carried out fairly and impartially.
    \item Candidates may be asked to complete an application form prior to interview.
    \begin{enumerate}
        \item The form must be made available to members at least five days before the deadline for submission.
        \item The form must not be excessive in length or effort required.
        \item The submissions must be considered by at least two members of the panel.
        \item The panel may choose to only interview a subset of the candidates based on their form responses.
    \end{enumerate}
    \item Where reasonably possible, the interview panel must remain consistent for all candidates for a specific appointment.
    \item The main interview questions must be the same for all candidates for a specific appointment.
    \begin{enumerate}
        \item This clause does not prohibit reasonable follow up questions in response to candidates' answers.
    \end{enumerate}
    \item Prior to the interview, candidates must be made aware of the criteria with which the interview panel intends to make its decision.
    \item Candidates have the right (which they may waive) to be given at least four days' notice for their interview.
    \item Before their interview and with permission of the Chair, candidates may be given a small project set by the panel to explain in their interview.
    \item Candidates who require it may have interview questions given in writing (during the interview), including any follow up questions.
    \item Interview questions must be created to give candidates an opportunity to give responses that best demonstrate their abilities relating to the criteria.
    \item The Chair must act as interview chair.
    \begin{enumerate}
        \item The Chair may send the Deputy Chair or an appropriate Executive Officer to act as interview chair on their behalf.
        \item The interview chair must not hold any other position on the interview panel or have an unfair conflict of interest.
    \end{enumerate}
    \item A candidate must not sit on the interview panel.
    \item For interviews for appointed officers (as listed in Committees and Job Descriptions) Committees and Job Descriptions:
    \begin{enumerate}
        \item The interview panel should be as close as reasonably possible to as follows:
        \begin{enumerate}[(a)]
            \item The interview chair,
            \item The President,
            \item Any relevant Executive Officers, \emph{and}
            \item The incumbent position holder.
        \end{enumerate}
        \item Notwithstanding clause \ref{preterm-prep}, the interview panels for Head of Ball, Head of Mildert Day and Head of VMCFS do not include the Events Officer-elect.
        \item The application form may be written by one or more of the panel and must be agreed by the whole panel.
        \item Candidates may submit a statement of intent describing their plans for the role. This statement must be no longer than one side of A4.
        \item Once a successful candidate is selected by the panel, the Chair must inform the JCR at the next meeting and on the JCR Facebook group.
        \item A member of the JCR may formally oppose the appointment of a candidate.
    \end{enumerate}
    \item For interviews for committees:
    \begin{enumerate}
        \item The interview panel should be as close as reasonably possible to as follows:
        \begin{enumerate}[(a)]
            \item The interview chair,
            \item The head of the committee (or deputy if this is not possible),
            \item Any relevant Executive Officers, \emph{and}
            \item Up to two committee members, at the discretion of the interview chair.
        \end{enumerate}
        \item The head of the committee must write the application form and the main interview questions which must be agreed by the whole panel.
    \end{enumerate}
    \item The interview chair must ensure that the appointment process is carried out fairly and in accordance with the Standing Orders.
    \item The interview chair may deem a question to be inappropriate and the candidate must not answer it.
    \item Following the interviews, the panel will discuss each of the candidates, comparing them to the criteria.
    \item Selections must be based only on the interviews, subsequent discussions by the panel and their written application/statement of intent.
    \item If the discussion is unsuccessful in determining the successful candidate(s) by consensus, then selection may be decided by secret ballot of the panel.
    \item If any positions are not filled after the initial interview process, then the interview process will re-open again to fill the remaining positions at a reasonable future date.
    \item If there is any evidence of improper behaviour or bias towards any candidate, whether immediately apparent or later reported, then the JCR Chair must report the matter to Governance Committee who may nullify the decisions of the interview process and order it to be restarted.
\end{enumerate}
\newpage

\section{Clubs and Societies}
\begin{enumerate}
    \item A club or society affiliated to the JCR may use the college name and apply for funding from the JCR.
    \item All clubs and societies affiliated with the JCR are bound by the Standing Orders.
    \item All clubs and societies must have a governing document.
    \begin{enumerate}
        \item The JCR Chair may make final binding rulings on matters of interpretation.
        \item The JCR Standing Orders take precedence.
        \item Clubs and societies must use the governing document template in Appendix L, unless otherwise approved by Governance Committee.
        \item A club or society may choose to have a supplementary constitution, which must be approved by Governance Committee.
        \item A club or society may be required to have a supplementary constitution by Governance Committee, where the club or soceity must be given a reason.
        \item All supplementary constitutions must be available to all JCR members.
    \end{enumerate}
    \item All clubs and societies must have a president, who is solely responsible for ensuring that the club or society's aims and commitments are fulfilled.
    \begin{enumerate}
        \item Unless otherwise stated in the JCR Standing Orders, executive committees must be elected in fair and free elections that may be observed by the JCR Chair and Governance Committee.
        \item A president may delegate day-to-day responsibilities to other officers, provided they are set out in a supplementary constitution.
    \end{enumerate}
    \item Any JCR committee, club or society must ensure that spaces they have used are returned to a useable state as soon as reasonably possible.
    \item All equipment and funds of all affiliated clubs and societies are property of the JCR.
    \item Any member of the JCR may propose the creation of a club or society to be affiliated to the JCR. The procedure is as follows:
    \begin{enumerate}
        \item The member must submit a proposal to the JCR Chair stating the purpose of the club/society and its aims and objectives.
        \item The Executive Committee must consider the proposal and decide if they will formally support or oppose it, giving reasons.
        \item The proposal must be presented to the next JCR meeting and approved by a referendum.
    \end{enumerate}
    \item A society may be designated as a Music Society by a motion of the JCR (including its creation motion).
    \item A club/society may not disaffiliate from the JCR.
    \item A club/society may be suspended or dissolved by a vote of the JCR.
    \item A club/society may be suspended by the JCR President if they are requested to do so by the University.
    \item In the event of suspension or dissolution, all equipment and funds of a club/society come under the control of the JCR Executive Committee.
\end{enumerate}
\newpage

\section{Committees}
\begin{enumerate}
    \item The quorum for all JCR committees is two-thirds of its membership, unless otherwise stated in the JCR Standing Orders or a higher quorum is stated in the committee constitution.
    \item The Chair (or their deputy) does not count for the purposes of committee quorums.
    \item \label{Chair Vote} The Chair cannot vote in committees except to give a casting vote in accordance with the following principles:
    \begin{enumerate}[(a)]
        \item The Chair should vote to enable further discussion or investigation.
        \item The Chair should vote to maintain the status quo where there is no majority in favour of change.
    \end{enumerate}
    \item Committees of the JCR may decide to have their own constitution which must be approved by Governance Committee.
    \begin{enumerate}
        \item The JCR Chair may make final binding rulings on matters of interpretation.
        \item The JCR Standing Orders take precedence.
        \item All constitutions must be available to all JCR members.
    \end{enumerate}
    \subsection{Membership}
    \item An open committee is one that any JCR member may join on request.
    \begin{enumerate}
        \item JCR committees are not open unless this is explicitly stated in the Standing Orders.
    \end{enumerate}
    \item For non-open committees, ordinary committee members must be appointed by interview unless otherwise stated in the Standing Orders.
    \item \label{vacant-quorum}A vacant position must still be counted for the purposes of determining if a committee is quorate.
    \item An ordinary committee member who does not attend at least 60\% of meetings in a term without reasonable excuse may be removed by the head of the committee with permission of the Chair.
    \item An ordinary committee member who does not satisfactorily fulfil their committee responsibilities or commits misconduct may be removed by the head of the committee with permission of Governance Committee.
    \item A member may appeal removal in accordance with the Permanent Standing Orders.
    \item Unless otherwise stated, the head of the committee has discretion as to whether to appoint a replacement of a vacating ordinary committee member.
    \subsection{Temporary committees}
    \item A Temporary Committee may be created by the Executive Committee with approval of a motion of a JCR meeting.
    \begin{enumerate}
        \item The motion must specify the purpose, the membership (including the head), and the lifetime of the committee, and if the committee will be able to spend JCR funds.
        \item Further motions may be passed to amend or disband the committee.
    \end{enumerate}
\end{enumerate}

\section{JCR Officers}
\begin{enumerate}
    \item An officer ceases to hold office if their position is abolished.
    \begin{enumerate}
            \item However, if an abolished position has a clear replacement, the incumbent may continue to serve in the new position for the remainder of their term of office.
    \end{enumerate}
    \item If an Officer resigns or is removed, then the Executive Committee must ensure that:
    \begin{enumerate}[(a)]
        \item the duties of their position are fulfilled by another Officer or Officers until the position is filled, \emph{and}
        \item a replacement is elected/appointed as soon as possible.
    \end{enumerate}
    \item\label{preterm-prep} Where necessary, a member who has been elected/appointed but does not yet hold office may act as reasonably necessary in preparation for their term of office.
    \begin{enumerate}
        \item This includes appointing officers and committee members.
    \end{enumerate}
    \item Officers must prepare a position card explaining their role and submit this to the Chair prior to the election/appointment process for their successor.
    \item Officers must prepare a handover document explaining how to carry out the role for their successor.
    \begin{enumerate}
        \item The handover document must be kept up-to-date throughout the term of office.
        \item Handover documents must be stored centrally in a system chosen by the President.
    \end{enumerate}
    \item An official position can only be held by one person.
    \begin{enumerate}
        \item Notwithstanding this clause, each of the positions of Head of a Talk and Support Committee can be held jointly by two JCR members.
    \end{enumerate}
\end{enumerate}
\newpage

\section{The Executive Committee}
\begin{enumerate}
    \item The Executive Committee, whenever possible, will be chaired by the Chair.
    \begin{enumerate}
        \item When this is not possible, they must appoint the Deputy Chair or any other member of the Executive Committee to chair in their place.
    \end{enumerate}
    \item The Executive Committee may delegate any of its powers to any sub-committee consisting of one or more Executive Officers.
    \begin{enumerate}
        \item Any such delegation is made subject to any conditions the Executive Committee may impose and may be revoked or altered at any time.
    \end{enumerate}
    \item Any Executive Officer may be mandated by the JCR to sit on a JCR committee not included in their normal responsibilities by a motion at a JCR meeting.
    \item Past and present Executive Officers/Officers-elect must not propose or second a nomination for any post or office of the JCR.
    \item The Executive Committee must meet at least twice a term.
    \item The quorum for the Executive Committee is 7 Officers.
    \item Meetings of the Executive Committee may normally be closed to ordinary JCR members.
    \item A pre-publicised open Executive Committee meeting must take place at least once per term.
    \item Complete and accurate minutes of all Executive Committee meetings must be made and published within a reasonable time frame.
    \item Executive Officers have the right to speak as an ordinary member (while retaining their Office) against a motion which the Executive Committee is supporting.
    \item Executive Officers in College for Executive Committee business may receive free meals, authorised by the President.
\end{enumerate}
\newpage

\section{Governance Committee}
\begin{enumerate}
    \item Governance Committee must meet formally at the start of every term.
    \item Governance Committee may exclude any of its members from a meeting or part of a meeting if reasonably necessary to successfully fulfil its scrutiny functions.
    \item For Governance Committee, the following special conditions apply to the quorum:
    \begin{enumerate}
        \item The quorum only counts non-Executive Officer members.
        \item The committee is considered to be inquorate if non-Executive Officer members do not outnumber the voting Executive Officer members.
        \item For the fulfilment of this requirement, an Executive Officer member may exclude themself from votes while remaining present at a meeting.
    \end{enumerate}
    \item Governance Committee meetings must be chaired by one of the non-Executive Officer members on a rotating basis.
    \item Voting in Governance Committee meetings must be anonymous.
    \item Governance Committee can require the attendance of any Officer at a meeting if this is reasonably necessary for the fulfilment of its functions.
    \item Procedure to allow Governance Committee to fulfil its scrutinising roles:
    \begin{enumerate}
        \item Any member of Governance Committee can call a meeting;
        \item The Assistant Head of Governance Committee (Deputy Chair) should report any instances of misconduct to both the Governance Committee and Executive Committee.
    \end{enumerate}
\end{enumerate}

\section{Finances}
\begin{enumerate}
    \item The JCR's financial year runs from the 1st August to 31st July.
    \item All committees, clubs and societies affiliated to the JCR that carry out trading activities must produce annual accounts of income and expenditure and submit these to the JCR FACSO for audit by the Financial Auditing Committee.
    \item All committees, clubs and societies affiliated to the JCR must have their accounts audited semi-annually by the Financial Auditing Committee.
    \item No JCR committee, club or society may use its funds for payment of fines nor may it collect money from its members for the purpose of paying possible future fines.
    \begin{enumerate}
        \item This clause may be waived in exceptional circumstances by the JCR Treasury Committee.
    \end{enumerate}
    \item All candidates for the positions of JCR President and JCR FACSO must be interviewed to determine their financial competence.
    \begin{enumerate}
        \item The interview must be solely about the financial aspects of the position.
        \item The interview panel must consist of:
        \begin{enumerate}[(a)]
            \item The JCR Chair, as chair of the interview
            \item The JCR President
            \item The JCR FACSO
            \item A member of Financial Auditing Committee.
        \end{enumerate}
        \item The interview panel must decide whether the candidate has the necessary understanding of the financial aspects of the position and report this to the hustings.
        \item The minutes of the interview must be made available to the JCR before the hustings.
    \end{enumerate}
    \subsection{The Annual Budget}
    \item The JCR FACSO must produce the Annual Budget which must show all predicted expenditure and income for the JCR account for the financial year.
    \item The JCR FACSO must present the Annual Budget to the Executive Committee and Treasury Committee before the first JCR meeting of the year.
    \begin{enumerate}
        \item The Annual Budget does not need approval of either committee.
        \item Members of either committee can suggest amendments, but it is at the discretion of the JCR FACSO to implement them.
        \item A majority of two thirds, on either committee, can reject the Annual Budget or parts of it.
    \end{enumerate}
    \item The JCR FACSO will present the Annual Budget at the first JCR meeting of the academic year corresponding to the relevant financial year and must be approved by a referendum following the JCR meeting.
    \begin{enumerate}
        \item The `academic year corresponding to the relevant financial year' is the academic year which overlaps most with the relevant financial year.
        \item If the Annual Budget is not approved, it may be provisionally implemented until the end of December with the support of both Treasury Committee and the Executive Committee. The FACSO must submit a revised Annual Budget at a subsequent JCR Meeting.
        \item If an Annual Budget has not been approved at the end of December:
        \begin{enumerate}[(a)]
            \item All expenditure must cease until an Annual Budget is approved;
            \item Spending that is crucial to the operation of the JCR must be identified by the Executive Committee and may be approved by the Board of Trustees.
        \end{enumerate}
    \end{enumerate}
    \subsection{Approval of Spending}
    \item Any expenditure that was part of the approved Annual Budget for that financial year does not require further approval.
    \item Expenditure on individual products/services up to £200 must be approved by the JCR President and FACSO together.
    \item Expenditure on individual products/services from £200 up to £1,000 must be approved by Treasury Committee and Executive Committee.
    \item Expenditure on individual products/services above £1,000 must be approved by Treasury Committee and Executive Committees and then by a referendum following a JCR meeting.
    \item If a financial motion in a JCR meeting is amended, then the amended motion must be approved by Treasury Committee at its next meeting.
    \begin{enumerate}
        \item If the expenditure increases, the process of approval must restart.
    \end{enumerate}
    \subsection{The JCR Levy}
    \item In their first year of study, new undergraduate members must pay the JCR levy (the ``full JCR levy'').
    \item A member of the MCR who opts into JCR membership must pay a levy (the ``postgraduate levy'') set by the JCR President and JCR FACSO in consultation with the MCR.
    \begin{enumerate}
        \item The postgraduate levy must not be more than 1/3 of the full JCR levy.
    \end{enumerate}
    \item A new undergraduate member who does not pay their JCR levy by the end of week 2 of their first term without reasonable excuse is considered to have opted out of JCR membership.
    \begin{enumerate}
        \item For the purposes of this clause, what constitutes a ``reasonable excuse'' is at the discretion of the FACSO.
    \end{enumerate}
    \item A student visiting as part of an international exchange programme may pay a reduced levy (the ``exchange student levy'') as set by the FACSO.
    \begin{enumerate}
        \item The exchange student levy must be set at a rate per academic term and must not be more than 1/9 of the full JCR levy.
    \end{enumerate}
    \item A non-member who wishes to join the JCR must pay a levy.
    \begin{enumerate}
        \item An undergraduate student must pay 1/3 of the full JCR levy for each year or part year that they have remaining of their degree, up to the full JCR levy amount.
        \item If a student previously paid the levy and subsequently opted out, then the amount they paid previously (less any refunded) must be taken into account.
    \end{enumerate}
    \item Non-members may be charged more than members for goods, services, events and JCR-affiliated club/society memberships where they are partially funded by JCR levy contributions.
    \begin{enumerate}
        \item The additional amount will be decided by the JCR FACSO.
        \item The FACSO is responsible for maintaining a list of current undergraduate students of the College who are not members of the JCR to facilitate the enforcement of this clause.
    \end{enumerate}
    \item To change the full JCR levy amount for the next academic year, a proposal must be submitted by the JCR FACSO to the Executive Committee and the Treasury Committee. If both committees approve the proposal, then it must be approved by a motion at a JCR meeting to come into effect.
    \item A new undergraduate member may be permitted to pay their JCR levy in three equal instalments under extenuating circumstances.
    \begin{enumerate}
        \item For the purpose of this clause, what constitutes ``extenuating circumstances'' is at the discretion of the FACSO.
        \item The three instalments must be collected in each of the three terms, Michaelmas, Epiphany, and Easter, respectively; the exact date of each payment will be set by the FACSO.
        \item An undergraduate member that fails to make payment of any of the instalments will be considered to have opted out of membership.
    \end{enumerate}
    \subsection{Treasury Committee}
    \item The minutes of Treasury Committee meetings must be made available to all members of the JCR without unreasonable delay.
    \item Any Officer may be required by the FACSO to attend specific meetings of the Treasury Committee if reasonably necessary.
    \item All events budgets must be brought before and approved by Treasury Committee before any spending.
    \begin{enumerate}
        \item Treasury Committee may suggest amendments to these budgets.
    \end{enumerate}
    \item Any expenditure which is rejected by Treasury Committee may be approved by a motion at a JCR meeting at which the FACSO must explain the reasons for the rejection.
    \item Meetings of Treasury Committee must be chaired by the Chair or Deputy Chair.
    \begin{enumerate}
        \item Where this is not possible, the Chair must appoint an Executive Officer, other than the FACSO or President, to chair in their place.
    \end{enumerate}
    \item Treasury Committee may continue to exist - remotely - between the end of Easter term and start of Michaelmas term as the ``Summer Treasury Committee''.
    \begin{enumerate}
        \item The ordinary committee members of the Summer Treasury Committee are those of the preceding Treasury Committee who are still JCR members.
        \item The Summer Treasury Committee advises and assists the FACSO with the creation of the annual budget.
    \end{enumerate}
\end{enumerate}

\section{Awards}
\begin{enumerate}
    \item At the end of Easter Term, the JCR will present awards to recognise the achievements and contributions of members.
    \subsection{Principal's Awards}
    \item Principal's Awards are awarded to finalists to recognise their contributions to the JCR and wider community.
    \item The President must take nominations from JCR members.
    \item The President must establish a panel representative of the JCR to select recipients.
    \begin{enumerate}
        \item The panel should (though is not required to) comprise a minimum of two students from each of the first three year groups and the President-elect.
        \item The Principal must be invited to join the panel.
        \item The panel must be chaired by one of the following (in order of precedence) who must not have been nominated for an award:
        \begin{enumerate}[(a)]
            \item Deputy Chair,
            \item Chair,
            \item A member of Governance Committee, \emph{or}
            \item An ordinary member of the JCR deemed trustworthy by the President.
        \end{enumerate}
    \end{enumerate}
    \item Members of the panel who are being considered for an award must not be present while recipients for the award are being decided.
    \begin{enumerate}
        \item Members who are selected as recipients of the award may return to the panel if the benefits of their presence would outweigh any potential conflict of interest.
    \end{enumerate}
    \item The Principal will present the awards at the Principal's Formal.
    \subsection{Sports Awards}
    \item Sports Awards are awarded to recognise contributions and achievements in sport.
    \item The Sports and Societies Officer must take nominations from JCR members.
    \item The Sports and Societies Officer must establish a panel representative of the JCR sports community to select recipients.
    \item The Principal will present the awards at the Sports Formal.
    \subsection{College Colours}
    \item College Colours are awarded to non-finalists to recognise their contributions to the JCR and wider community.
    \item First year students may be awarded Half Colours.
    \item Second year students may be awarded Full colours.
    \item The Executive Committee must collectively select recipients.
    \item Each Executive Officer may nominate a candidate for Half Colours and a candidate for Full Colours.
\end{enumerate}


\end{document}