\documentclass[12pt]{article}  % Set font size to 12; article is default LaTeX doc type
% \usepackage{helvet}  % Use Helvetica
\usepackage[utf8]{inputenc}  % Set encoding; LaTeX default
\usepackage[hidelinks]{hyperref}  % Table of contents links
\renewcommand{\familydefault}{\sfdefault}  % Use sans serif font
\usepackage{graphicx}  % To include coat of arms
\graphicspath{ {./} }  % Graphics in same folder as this file
\setlength{\parindent}{0pt}  % Disable indentation at start of paragraphs
\setlength{\parskip}{1em}  % Paragraph spacing
\usepackage[margin=1.2in]{geometry}  % Margins
\renewcommand{\arraystretch}{1.5}  % Table padding
\hyphenpenalty=10000  % Prevent hyphenation
\emergencystretch \textwidth  % Fix spacing issues caused by lack of hyphenation
% Set up numbering
\usepackage{enumerate}
\renewcommand{\labelenumi}{\theenumi}
\renewcommand{\theenumi}{\thesection.\arabic{enumi}.}  % Section number followed by item number for clauses
\renewcommand{\labelenumii}{\theenumii}
\renewcommand{\theenumii}{\theenumi\arabic{enumii}.}  % Subclauses
% Disable subsection numbering
\usepackage{titlesec}
\renewcommand{\thesubsection}{}
\titleformat{\subsection}{\normalfont\large\bfseries}{}{0pt}{}


\title{Operational Standing Orders}
\author{Van~Mildert~College Junior~Common~Room}
\date{26th October 2018}

\begin{document}
\begin{titlepage}  % Title page
\maketitle
\begin{figure}[h]
\includegraphics[scale=0.25]{arms}  % Coat of arms
\centering
\end{figure}
\thispagestyle{empty}
\end{titlepage}

\setcounter{page}{2}  % Correct page numbering
\section*{Introduction}
These are the Operational Standing Orders of Van Mildert College Junior Common Room, approved on 26th October 2018.

This version was written by the Steering Committee of 2017-18:\\
\hspace*{2cm}Georgina Robinson (JCR Chair) (HLM)\\
\hspace*{2cm}Alastair Hargreaves-McManus\\
\hspace*{2cm}George Singleton\\
\hspace*{2cm}James Smith\\
\hspace*{2cm}Amelia Spencer (JCR Vice-Chair)

Amended in October 2018 by the Executive Committee, Sheehan Quirke (JCR President) and Alastair Hargreaves-McManus.
\newpage
\tableofcontents{}
\newpage


\section{Meetings}
\begin{enumerate}
    \subsection{The Chair}
	\item Where this section (‘Meetings’) refers to the “Chair” and not the “JCR Chair” explicitly, then it refers to the member chairing the meeting as necessary.
	\item The JCR Chair must chair all meetings unless they are absent or feel that it is inappropriate.
	\item The Chair must step down during an item of business in which they have a personal interest.
	\begin{enumerate}
		\item This clause may be waived if it would cause unreasonable disruption.
	\end{enumerate}
	\item If the JCR Chair is unable to chair a meeting or item of business, then one of the following officers must chair it (in order of precedence):
	\begin{enumerate}[(a)]
		\item The JCR Vice-Chair
		\item The JCR President
		\item The JCR Vice-President (Development)
		\item The JCR Vice-President (Welfare)
		\item The Communications Officer
		\item Any other member of the Executive Committee
	\end{enumerate}
	\subsection{Conduct and Discipline}
	\item The Chair must act impartially and in accordance with the Standing Orders.
	\item The Chair takes precedence over all other speakers.
	\item The Chair chooses who may speak at any given time.
	\begin{enumerate}
		\item The Chair must, where reasonable, give preference to those who have not spoken already on the matter under debate.
	\end{enumerate}
	\item The Chair may expel any member who is causing unreasonable disruption to the meeting.
	\item All members must behave in a respectful and courteous manner during meetings.
	\item All speeches must be relevant to the current business.
	\subsection{Attendance}
	\item All members of the JCR have the right to attend and speak at JCR meetings.
	\item Any person who is not a member of the JCR may attend and speak at a JCR meeting with permission of the Chair when it does not influence JCR business; for example, when candidates for roles in the Students’ Union canvass at a JCR meeting.
	\subsection{The Agenda}
	\item An agenda must be prepared for the meeting describing the planned order of business.
	\begin{enumerate}
		\item The content of the agenda is decided by the Chair.
		\item All matters to be included on the agenda for a meeting must be sent to the Chair.
		\item The agenda must conclude with “Any Other Business”.
	\end{enumerate}
	\subsection{Motions}
	\item All motions must be supported by two members of the JCR or by a recognised JCR committee, club or society.
	\item A motion that does not appear on the agenda may be brought under “Any Other Business” and discussed at the discretion of the Chair, except for the following types of motion:
	\begin{enumerate}[(a)]
		\item A motion to authorise the use of JCR funds.
		\item A motion to amend the Standing Orders.
		\item A motion of no confidence.
		\item A motion to give a formal warning.
	\end{enumerate}
	\item A procedural motion, as defined by the relevant clauses below, may be proposed at any time.
	\item Any member of the JCR may propose an amendment to a motion. If this is seconded by another member the meeting will debate the amendment and put it to a vote.
	\begin{enumerate}
		\item An amendment may not be amended. The original amendment must be put to a vote before a second amendment can be discussed.
	\end{enumerate}
	\item The Chair may amend motions without vote to correct a printing or grammatical error only.
	\item A motion of no confidence can only be moved by the JCR President, Constitutional Committee or a member with the supporting signatures of at least 10\% of JCR members.
	\item Motions of no confidence and motions to give formal warnings must receive at least two-thirds of the vote to be passed.
	\subsection{General Procedure}
	\item Only one matter at a time may be discussed.
	\item Any member of the JCR may raise a point of order which must take precedence over all other business except when a speech is being delivered.
	\begin{enumerate}
		\item A point of order raised during a debate must relate to the conduct of the meeting at that time.
	\end{enumerate}
	\item Any member of the JCR may raise a point of information to offer strictly factual information relating to the current business.
	\begin{enumerate}
		\item A speaker may choose to accept a point of information immediately or direct it to be heard at the end of their speech.
		\item The Chair may refuse to allow further points of information if they are being used inappropriately.
	\end{enumerate}
	\item For all motions unless the Standing Orders specify otherwise, the procedure is as described in this clause:
	\begin{enumerate}
		\item The proposer (or their representative) must give a speech explaining the motion. Members may then ask questions about the motion.
		\item If a member offers formal opposition then they may give a speech against the motion. If multiple members offer formal opposition, then the Chair shall choose the member that appears to have the strongest support from those affected by the motion in question.
		\item The motion will then be debated by the meeting. This debate will last as long as the Chair feels is reasonable.
		\item During the debate, amendments to the motion may be proposed and debated in accordance with the relevant clauses above.
		\item Following the debate, the proposer (or their representative) and the formal opposer may give speeches of summation.
		\item Finally, the motion shall be put to a vote of the meeting or put to a referendum if required by the Standing Orders or at the discretion of the Chair.
	\end{enumerate}
	\subsection{Procedural Motions}
	\item A procedural motion is defined as one that falls into one of the following categories:
	\begin{enumerate}[(a)]
		\item A motion to challenge a ruling of the Chair
		\item A motion to remove an item from the agenda
		\item A motion to hold an item until the next meeting
		\item A motion to end the current debate
		\item A motion to take the motion under debate in parts
		\item A motion to order that the current motion be put to a referendum
		\item Any other motion at the discretion of the Chair
	\end{enumerate}
	\item A procedural motion cannot go to a referendum and must be discussed within the meeting itself. For the motion to pass, it must meet a quota of 50\% of the votes.
	\item Any member of the JCR may propose a procedural motion which must be seconded by another member. The motion must be relevant to the current business.
	\item A procedural motion takes precedence over any other business. If more than one procedural motion is moved, then their order of precedence is that of the list of categories in the clause defining procedural motions above.
	\item A procedural motion is binding unless it contradicts the Standing Orders.
	\item If a procedural motion is moved to challenge a ruling of the Chair, then the procedure is as follows:
	\begin{enumerate}
		\item The Chair will be deemed automatically to have a personal interest in the debate and must act accordingly.
		\item The proposer must give a short speech setting out their challenge and the (challenged) Chair may give a speech setting out their defence.
		\item The motion will then be voted on by the meeting or withdrawn at the choice of the proposer.
		\item The motion will be approved only if it receives a two-thirds majority of the vote.
		\item If the motion is approved, then the challenged ruling is invalid and the (challenged) Chair must vacate the Chair for the rest of the item of business relating to the challenged ruling.
	\end{enumerate}
	\subsection{Hustings}
	\item The procedure for hustings is as described in this clause:
	\begin{enumerate}
		\item The order of candidates in the hustings shall be decided randomly by the Chair.
		\item Each candidate’s proposer shall give a one-minute speech followed by the candidate giving a two-minute speech.
		\item Each candidate must sing a song, perform a poem, or tell a joke. This may be done with the proposer and/or seconder of the candidate. This may be done before or after their speeches; order to be discussed with the Chair prior to the meeting.
		\item In the case of Presidential hustings, each candidate's campaign video (if approved) will then be played to the meeting.
		\item Each candidate may then each ask one question.
		\item The incumbent may then ask questions. The incumbent must act impartially.
		\item Questions will then be taken from the floor. Members of the Executive Committee may ask questions before other members.
		\item Candidates may then ask further questions at the discretion of the Chair.
		\item In the case of Presidential hustings, each candidate may finally make some concise final comments.
		\item Finally, the statement for RON (if one exists) is read by the Chair or the permitted member.
	\end{enumerate}
	\item Questions may be addressed at individual candidates or at all candidates.
	\item Length of answers permitted are to be determined at the discretion of the Chair.
	\item In the event of a question addressed at an individual candidate, the other candidates may also make a comment in response.
	\item Any hustings must have been on the agenda at the start of the meeting to be valid.
\end{enumerate}
\newpage

\section{Elections and Referenda}
\begin{enumerate}
	\item For the purposes of this section (‘Elections and Referenda’), the word “ballot” shall be taken as referring to both elections and referenda of the JCR.
	\item For the purposes of this section (‘Elections and Referenda’), where “notice” is required to be given, this must be given by an email to all JCR members and by a post on the JCR Facebook group.
	\item For every vote of the JCR, the roles of Senior and Junior Returning Officers must be filled by two of the following Officers (in order of precedence):
	\begin{enumerate}[(a)]
		\item The Chair
		\item The Vice-Chair
		\item The President
		\item The Senior DSU Representative
		\item Any other member of the Executive Committee
	\end{enumerate}
	\item The Returning Officers are responsible for ensuring that procedures for all ballots are carried out correctly.
	\item The Returning Officers must act impartially and transparently.
	\item A candidate must not act as a Returning Officer for the same election.
	\item The voting for all ballots must be conducted using the online voting system provided by the University.
	\begin{enumerate}
		\item If a member is unable to vote using this method, then they may make their vote by signed letter or email from their official University account sent to the Senior Returning Officer. 
	\end{enumerate}
	\item Voting for a ballot will be open for a period set by the Senior Returning Officer which must be no shorter than 24 hours and no longer than 168 hours (7 days).
	\begin{enumerate}
		\item Notice of the times of the start and end of the period must be given at least 48 hours before the start of voting.
		\item Notice must be also given at the start of the voting period.
	\end{enumerate}
	\item At appropriate times during the voting period, a voting station comprising at least device capable of accessing the online voting system must be set up in a prominent position within the main college building and overseen by a Returning Officer or by a member of the Executive Committee.
	\begin{enumerate}
		\item Candidates or interested parties must not loiter near the voting station.
		\item The manifestos of all candidates must be displayed at the voting station.
	\end{enumerate}
	\item At the end of the voting period, the Returning Officers must promptly announce the results of the ballot, including releasing the results on the online voting system.
	\item A person must not cause a change to the results of a ballot by any action other than the casting of a single lawful vote.
	\begin{enumerate}
		\item Consequentially:
		\begin{enumerate}[(a)]
			\item A member must not vote more than once in the same ballot.
			\item The results of a ballot must not be altered by any JCR Officer or Committee.
		\end{enumerate}
	\end{enumerate}
	\item A member must not reveal how another member has voted.
	\item To maintain the secrecy of the ballot, a member must not prove to any person how they have voted. 
	\item If the Returning Officers have reasonable suspicion that the Standing Orders have been contravened regarding a ballot, then they must refer the matter to Constitutional Committee.
	\item A ballot may be declared invalid if either the Returning Officers or Constitutional Committee conclude that the Standing Orders have been contravened regarding a ballot and that it is in the interests of democracy to rerun the ballot.
	\item If a ballot is declared invalid, then—
	\begin{enumerate}[(a)]
		\item The result is void.
		\item Constitutional Committee must decide whether the whole ballot process must be restarted or whether only the voting process must be rerun.
		\item The Returning Officers must act in accordance with the decision made by Constitutional Committee within a reasonable timeframe.
	\end{enumerate}
	\subsection{Referenda}
	\item A referendum must be held within three weeks of the relevant JCR meeting.
	\begin{enumerate}
		\item The Chair is responsible for ensuring that this clause is adhered to.
	\end{enumerate}
	\item For all referenda, a copy of the motion together with all other relevant information must be included when notice is given (in accordance with the Permanent Standing Orders) and must be available on the online voting system.
	\subsection{Elections}
	\item Nominations for elections must be submitted to the Senior Returning Officer in the time period and manner specified by the Chair.
	\begin{enumerate}
		\item The time period specified must be no shorter than five days unless Constitutional Committee concludes that there are extraordinary circumstances that make a shorter period necessary for the good governance of the JCR.
		\item The manner specified must be reasonable and must not discriminate against any potential candidates.
		\item The Senior Returning Officer must give notice of the time period and manner of submission at least one week before the end of the time period.
		\item Nominations must contain the names of a proposer and a seconder who are members of the JCR and not members of the Executive Committee.
		\item Nominations must be accompanied by a manifesto not exceeding one side of A4 paper. Candidates may also submit a plain text version of their manifesto for accessibility.
		\item Nominations for a sabbatical role must also be accompanied by a policy statement not exceeding one side of A4 paper.
	\end{enumerate}
	\item Candidates may also submit a 200 word statement for display on the online voting system to the Senior Returning Officer before the start of voting.
	\item Unless otherwise stated in the Standing Orders, only sole individuals may be elected to a position.
	\item All candidates standing for election to be President of Vice-President (Welfare) must meet with College Officers before submitting their nomination.
	\begin{enumerate}
		\item A candidate may be deemed unsuitable by the College Officers if they have reasonable grounds to believe that the election of that person would not be in the best interests of either the candidate’s personal welfare of that of the wider JCR or if they conclude that the person is wholly unsuitable for the position. The College Officers are expected to justify such a decision to the candidate and the JCR. Such a candidate is then ineligible to stand at that election. 
	\end{enumerate}
	\item Candidates must take part in hustings or are be deemed to have withdrawn from the election.
	\begin{enumerate}
		\item Candidates may take part in hustings remotely by video call, or similar, if their physical attendance is not possible.
		\item The Senior Returning Officer may waive this clause at their discretion in the event of exceptional circumstances.
	\end{enumerate}
	\item Accurate minutes of the hustings must be taken and published on the JCR Facebook group before the start of voting.
	\item All elections will be conducted by the Single Transferable Vote system in accordance with the rules set out by the Electoral Reform Society.
	\item In an uncontested election, RON is considered to have won if it receives over 33\% of the vote.
	\item In a contested election, RON will be treated as a candidate for the purposes of voting procedures.
	\item The election process will be restarted at a reasonable future date set by the Chair in consultation with Constitutional Committee and the Executive Committee, in the event of any of the following occurrences:
	\begin{enumerate}[(a)]
		\item No valid nominations are received
		\item The election is inquorate
		\item RON wins the election
	\end{enumerate}
	\subsection{Campaigning and Publicity}
	\item In elections for Executive Officers, candidates may canvass on up to three days before the opening of voting takes place. Canvassing must be monitored by a member of Constitutional Committee. The maximum total time spent canvassing is 4.5 hours.
	\begin{enumerate}
		\item Canvassing should occur in three 90-minute slots.
		\item This can be waived if Constitutional Committee allow.
	\end{enumerate}
	\item In elections for Non-Executive Officers, candidates must not canvass.
	\item All approved manifestos will be printed by the Senior Returning Officer.
	\begin{enumerate}
		\item The same number of manifestos must be printed for each candidate.
		\item Candidates must only place their manifestos in the communal toilets of the JCR and residential blocks, excluding Deerness.
	\end{enumerate}
	\item All approved manifestos will be posted on the JCR Facebook group by the Senior Returning Officer after the closing of nominations.
	\item Candidates husting for the role of President must produce a three-minute campaign video to be approved by Constitutional Committee before it is shown at the hustings.
	\begin{enumerate}
		\item Once shown at the hustings, campaign videos may be published online.
	\end{enumerate}
	\item In an election, a member may only campaign on behalf of RON with permission of the Senior Returning Officer.
	\begin{enumerate}
		\item The Senior Returning Officer should give permission to the first member who requests it.
		\item No more than one member can be given permission.
		\item A request to campaign on behalf of RON must be made before the day voting starts.
		\item Individuals conducting a campaign on behalf of RON have the right to complete anonymity, if requested.
		\item The individual representing RON must submit a statement, not exceeding one side of A4, to be read by the Senior Returning Officer at the end of the hustings. The individual must not take part in the hustings.
		\item The individual representing RON must not canvass.
		\item Unless otherwise stated in the Standing Orders, a member campaigning for RON is subject to the same election regulations as any other candidate. This includes campaign material entitlements, with the hustings statement acting as a manifesto.
	\end{enumerate}
	\item Campaigning on behalf of candidates or potential candidates for election is forbidden.
\end{enumerate}
\newpage

\section{Appointment by Interview}
\begin{enumerate}
	\item The following positions are appointed by interview:
	\begin{enumerate}[(a)]
		\item Assistant Events Officer (External)
		\item Assistant Events Officer (Internal)
		\item Carers Respite Committee Director
		\item Community Visiting Scheme Director
		\item DUCKtator
		\item Environmental Conservation Committee Director
		\item Head of Constitutional Committee
		\item Head of Decorations Committee
		\item Head of Gym Committee
		\item Head of Talk and Support Campaigns Committee
		\item Head of Talk and Support Pastoral Committee
		\item Primary School Project Director
		\item Secure Centre Mentoring Scheme Director
		\item Young Persons Project Director
	\end{enumerate}
	\item Ordinary members of committees must be appointed by interview unless otherwise stated in the Standing Orders.
	\item The process of appointment by interview must be carried out fairly and impartially.
	\item Where reasonably possible, the interview panel must remain consistent for all candidates for a specific appointment.
	\item The main interview questions must be the same for all candidates for a specific appointment.
	\begin{enumerate}
		\item This clause does not prohibit reasonable follow up questions in response to candidates’ answers.
	\end{enumerate}
	\item Prior to the interview, candidates must be made aware of the criteria with which the interview panel intends to make its decision.
	\item Candidates who require it may have interview questions given in writing (during the interview), including any follow up questions.
	\item Interview questions must be created to give candidates an opportunity to give responses that best demonstrate their abilities relating to the criteria.
	\item For interviews for positions listed in clause 3.1:
	\begin{enumerate}
		\item The interview panel must include the JCR Chair or Vice-Chair, who will chair the interview.
		\item The interview panel must include any relevant members of the Executive Committee.
		\item The interview panel must include the incumbent if this is possible. The incumbent must not sit on the panel if they are re-applying for the role.
		\item Candidates may submit a statement of intent describing their plans for the role. This statement must be no longer than one side of A4.
		\item Once a successful candidate is selected by the panel, the chair must inform the JCR at the next meeting and on the JCR Facebook group.
		\item A member of the JCR may formally oppose the appointment of a candidate.
	\end{enumerate}
	\item For interviews for committees:
	\begin{enumerate}
		\item The interview panel must include:
		\begin{enumerate}[(a)]
			\item The JCR Chair or Vice-Chair, who will chair the interview
			\item The head of the committee (or deputy if this is not possible)
			\item The JCR President or a JCR Vice-President
			\item  Up to two committee members, at the discretion of the interview chair
		\end{enumerate}
		\item Candidates must complete a written application form written by the head of the committee in consultation with the chair.
		\item The form must not exceed one side of A4 and must contain:
		\begin{enumerate}[(a)]
			\item A section for the candidate’s personal details,
			\item A section for the candidate to describe their experience, and
			\item (At most) three further relevant questions.
		\end{enumerate}
		\item The form must be made available to members at least five days before the deadline for submission.
		\item The written applications must be considered jointly by the head of the committee, the chair and one other member of the panel, and they must decide who will be invited for interview. There must be no upper or lower limit on the number of candidates invited for interview.
		\item Candidates must be given at least four days’ notice for their interview.
		\item The head of the committee must write the main interview questions and submit them to the rest of the panel at least five days before the first interview. 
		\item Before their interview and with permission of the chair, candidates may be given a small project set by the panel to explain in their interview.
	\end{enumerate}
	\item The chair of an interview must ensure that it is carried out fairly and in accordance with the Standing Orders. The chair may deem a question to be inappropriate and the candidate must not answer it.
	\item Following the interviews, the panel will discuss each of the candidates, comparing them to the criteria.
	\item Selections must be based only on the interviews, subsequent discussions by the panel and their written application/statement of intent.
	\item If the discussion is unsuccessful in determining the successful candidate(s) by consensus, then selection may be decided by secret ballot of the panel.
	\begin{enumerate}
		\item In event of three successive ties in such a ballot, the leading candidates must be interviewed for a second time. This may be by a different panel, at the discretion of the JCR Chair.
	\end{enumerate}
	\item If any positions are not filled after the initial interview process, then the interview process will re-open again to fill the remaining positions at a reasonable future date.
	\item If there is any evidence of improper behaviour or bias towards any candidate, whether immediately apparent or later reported, then the JCR Chair must report the matter to Constitutional Committee who may nullify the decisions of the interview process and order it to be restarted.
\end{enumerate}
\newpage

\section{Committees, Clubs and Societies}
\begin{enumerate}
	\item A club or society affiliated to the JCR may use the college name and apply for funding from the JCR.
	\item All clubs and societies affiliated with the JCR are bound by the Standing Orders.
	\item All clubs and societies must have a constitution of their own which must have been approved by Constitutional Committee. Committees of the JCR may also decide to have their own constitution which must be approved by Constitutional Committee.
	\begin{enumerate}
		\item The JCR Chair may make final binding rulings on matters of interpretation. 
		\item The JCR Standing Orders take precedence.
		\item All constitutions must be available to all JCR members.
	\end{enumerate}
	\item All clubs and societies must have an executive committee consisting of at least:
	\begin{enumerate}[(a)]
		\item Their head (President/Chair/Captain/Director etc.)
		\item Their Secretary
		\item Their Treasurer
	\end{enumerate}
	\begin{enumerate}
		\item Executive committees must be elected in fair and free elections that may be observed by the JCR Chair and Constitutional Committee.
	\end{enumerate}
	\item All JCR committee heads must submit a report to the JCR at the final meeting of the year.
	\item The quorum for all JCR committees is two-thirds of its membership, unless otherwise stated in the JCR Standing Orders.
	\item Any JCR committee, club or society must ensure that spaces they have used are returned to a useable state as soon as reasonably possible.
	\item All equipment and funds of all affiliated clubs and societies are property of the JCR.
	\item Any member of the JCR may propose the creation of a club or society to be affiliated to the JCR. The procedure is as follows:
	\begin{enumerate}
		\item The member must submit a proposal to the JCR Chair stating the purpose of the club/society and its aims and objectives.
		\item The Executive Committee must consider the proposal and decide if they will formally support or oppose it, giving reasons.
		\item The proposal must be presented to the next JCR meeting (as a motion) and approved by a vote of the JCR.
	\end{enumerate}
	\item A Temporary Committee may be created by the Executive Committee with approval of a motion of a JCR meeting.
	\begin{enumerate}
		\item The motion must specify the purpose, the membership (including the head), and the lifetime of the committee, and if the committee will be able to spend JCR funds.
		\item Further motions may be passed to amend or disband the committee.
	\end{enumerate}
	\item A club/society may be suspended or dissolved by a vote of the JCR.
	\item A club/society may be suspended by the JCR President if they are requested to do so by the University.
	\item In the event of suspension or dissolved, all equipment and funds of a club/society come under the control of the JCR Executive Committee.
\end{enumerate}
\newpage

\section{JCR Officers and Other Officials}
\begin{enumerate}
	\item A JCR Officer is defined as a member of the JCR elected to a post by an election where the electorate is the whole JCR as outlined in Section 2 of the Operational Standing Orders.
	\item An ordinary committee member is defined as being a member of a committee who is not a member ex officio.
	\item Where this section (‘JCR Officers and Other Officials’) uses the term “Other Official” it refers to a member of the JCR who holds an official position that is not elected and is not ordinary membership of a committee.
	\item Where this section (‘JCR Officers and Other Official’) refers to appointment, it refers to the standard method of appointment for that position be that election or interview or otherwise.
	\item All JCR Officers and Other Officials are accountable to the JCR.
	\item Unless otherwise stated in the Standing Orders, all Officers hold office for one year following their election.
	\item If a JCR Officer or Other Official resigns or is removed, then the Executive Committee must ensure that the duties of their position are fulfilled by another JCR Officer or Officers until a replacement is appointed.
	\item If JCR Officer, Other Official or ordinary committee member resigns or is removed, then a replacement must be appointed at the earliest possible time.
	\item A vacant position must still be counted for the purposes of determining if a committee is quorate.
	\item Members of any JCR committee who do not attend at least 60\% of committee meetings in a term are deemed to have resigned.
	\begin{enumerate}
		\item This clause may be waived at the discretion of the JCR Chair or the head of the committee if there are extenuating circumstances.
	\end{enumerate}
	\item Clauses 7.7 and 7.8 of the Permanent Standing Orders also apply to Other Officials in the same way that they apply to JCR Officers.
	\item If the JCR President vacates office, then the JCR may elect such Officers as the JCR deems necessary to assist the replacement non-sabbatical President.
	\item Following unsatisfactory fulfilment of committee responsibilities or misconduct, an ordinary committee member may be removed by the committee head with permission of Constitutional Committee.
	\item A member may appeal to the JCR Chair against removal from a committee. If the JCR Chair cannot resolve the issue satisfactorily then the member will have the write to request a hearing.
	\item The panel for an appeal hearing (for the purposes of the above clause) must be as follows:
	\begin{enumerate}[(a)]
		\item JCR President
		\item JCR Vice-Presidents
		\item The relevant committee head
		\item Another committee head
		\item An ordinary member of the relevant committee.
	\end{enumerate}
	\begin{enumerate}
		\item The panel must consider whether all necessary evidence had been considered before the removal decision was made.
		\item The panel may overrule the decision to remove the member if they feel that the removal was inappropriate.
		\item The panel itself can only be overruled by a two-thirds majority vote at a JCR meeting.
	\end{enumerate}
	\item In the event of misconduct by Other Officials, they can be warned or removed using the same procedures as for JCR Officers.
\end{enumerate}
\newpage

\section{Finances}
\begin{enumerate}
	\item The JCR’s financial year runs from the 1st August to 31st July.
	\item All committees, clubs and societies affiliated to the JCR that carry out trading activities must produce annual accounts of income and expenditure and submit these to the JCR Treasurer for audit by the Financial Auditing Committee.
	\item All committees, clubs and societies affiliated to the JCR must have their accounts audited semi-annually by the Financial Auditing Committee.
	\item No JCR committee, club or society may use its funds for payment of fines nor may it collect money from its members for the purpose of paying possible future fines.
	\begin{enumerate}
		\item This clause may be waived in exceptional circumstances by the JCR Treasury Committee.
	\end{enumerate}
	\item All candidates for the positions of JCR President and JCR Treasurer must be interviewed to determine their financial competence.
	\begin{enumerate}
		\item The interview must be solely about the financial aspects of the position.
		\item The interview panel must consist of:
		\begin{enumerate}[(a)]
			\item The JCR Chair, as chair of the interview
			\item The JCR President
			\item The JCR Treasurer
			\item A member of Financial Auditing Committee.
		\end{enumerate}
		\item The interview panel must decide whether the candidate has the necessary understanding of the financial aspects of the position and report this to the hustings.
		\item The minutes of the interview must be made available to the JCR before the hustings.
	\end{enumerate}
	\subsection{The Annual Budget}
	\item The JCR Treasurer must produce the Annual Budget which must show all predicted expenditure and income for the JCR account for the financial year.
	\item The JCR Treasurer will present the Annual Budget at the first JCR meeting of the academic year corresponding to the relevant financial year and must be approved by a motion at a JCR meeting.
	\begin{enumerate}
		\item The ‘academic year corresponding to the relevant financial year’ is the academic year which overlaps most with the relevant financial year.
		\item If the Annual Budget is not approved, it may be provisionally implemented until the end of December with the support of both Treasury Committee and the Executive Committee. The Treasurer must submit a revised Annual Budget at a subsequent JCR Meeting.
		\item If an Annual Budget has not been approved at the end of December, then all expenditure must cease until an Annual Budget is approved.
	\end{enumerate}
	\subsection{Approval of Spending}
	\item Expenditure on individual products/services up to £200 must be approved by the JCR President and Treasurer together.
	\item Expenditure on individual products/services from £200 up to £1,000 must be approved by Treasury Committee and Executive Committee.
	\item Expenditure on individual products/services above £1,000 must be approved by Treasury Committee and Executive Committees and then by a motion at a JCR meeting.
	\subsection{The JCR Levy}
	\item In their first year of study, new undergraduate members will be charged the JCR levy for three years of study (the ‘full’ JCR levy). 
	\item For each year of study after their first three, undergraduate JCR members must pay an additional fee equal to 1/3 of the full JCR levy (as their levy) or opt-out from membership
	\item A member of the MCR who opts into JCR membership must pay a fee (as their levy) set by the JCR President and JCR Treasurer in consultation with the MCR.
	\item If a student who has opted out of the JCR wishes to join the JCR and they have not previously paid the JCR levy, then they must pay a fee equal to 1/3 of the full JCR levy per year or part year that they are a member of the JCR (as their levy).
	\item Non-members may be charged more than JCR members for JCR events and JCR affiliated club/society memberships where they are partially funded by JCR levy contributions.
	\begin{enumerate}
		\item The additional amount will be decided by the JCR Treasurer.
		\item The President is responsible for maintaining a list of current undergraduate students of the College who are not members of the JCR to facilitate the enforcement of this clause.
	\end{enumerate}
	\item To change the levy amount for the next academic year, a proposal must be submitted by the JCR Treasurer to the Executive Committee and the Treasury Committee. If both committees approve the proposal, then it must be approved by a motion at a JCR meeting to come into effect.
\end{enumerate}
\end{document}