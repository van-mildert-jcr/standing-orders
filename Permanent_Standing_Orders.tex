\documentclass[12pt]{article}  % Set font size to 12; article is default LaTeX doc type
% \usepackage{helvet}  % Use Helvetica
\usepackage[utf8]{inputenc}  % Set encoding; LaTeX default
\usepackage[hidelinks]{hyperref}  % Table of contents links
\renewcommand{\familydefault}{\sfdefault}  % Use sans serif font
\usepackage{graphicx}  % To include coat of arms
\graphicspath{ {./} }  % Graphics in same folder as this file
\setlength{\parindent}{0pt}  % Disable indentation at start of paragraphs
\setlength{\parskip}{1em}  % Paragraph spacing
\usepackage[margin=1.2in]{geometry}  % Margins
\renewcommand{\arraystretch}{1.5}  % Table padding
\hyphenpenalty=10000  % Prevent hyphenation
\emergencystretch \textwidth  % Fix spacing issues caused by lack of hyphenation
% Set up numbering
\usepackage{enumerate}
\renewcommand{\labelenumi}{\theenumi}
\renewcommand{\theenumi}{\thesection.\arabic{enumi}.}  % Section number followed by item number for clauses
\renewcommand{\labelenumii}{\theenumii}
\renewcommand{\theenumii}{\theenumi\arabic{enumii}.}  % Subclauses
% Disable subsection numbering
\usepackage{titlesec}
\renewcommand{\thesubsection}{}
\titleformat{\subsection}{\normalfont\large\bfseries}{}{0pt}{}


\title{Permanent Standing Orders}
\author{Van~Mildert~College Junior~Common~Room}
\date{26th October 2018}

\begin{document}
\begin{titlepage}  % Title page
\maketitle
\begin{figure}[h]
\includegraphics[scale=0.25]{arms}  % Coat of arms
\centering
\end{figure}
\thispagestyle{empty}
\end{titlepage}

\setcounter{page}{2}  % Correct page numbering
\section*{Introduction}
These are the Permanent Standing Orders of Van Mildert College Junior Common Room, approved on 26th October 2018.

This version was revised by the Steering Committee of 2017-18:\\
\hspace*{2cm}Georgina Robinson (JCR Chair) (HLM)\\
\hspace*{2cm}Alastair Hargreaves-McManus\\
\hspace*{2cm}George Singleton\\
\hspace*{2cm}James Smith\\
\hspace*{2cm}Amelia Spencer (JCR Vice-Chair)

Amended in October 2018 by the Executive Committee, Sheehan Quirke (JCR President) and Alastair Hargreaves-McManus.
\newpage
\tableofcontents{}
\newpage


\section{Preface}
\begin{enumerate}
    \item In these Standing Orders the following terms shall be understood with the following meanings:\\
\begin{tabular}{|p{1.5in}|p{3.8in}|}
\hline
\textbf{Term}   & \textbf{Definition}\\ \hline
JCR             & Junior Common Room of Van Mildert College\\ \hline
MCR             & Middle Common Room of Van Mildert College\\ \hline
College         & Van Mildert College\\ \hline
College Council & College Council of Van Mildert College\\ \hline
DSU             & Durham Students' Union\\ \hline
Standing Orders & Collectively the Permanent Standing Orders, the Operational Standing Orders and Committees and Job Descriptions Standing Orders \\ \hline
Framework       & The DSO Framework\\ \hline
\end{tabular}
    \item The JCR is a student organisation which is part of the Durham Student Organisation (DSO) and operates within the DSO Framework. The JCR voted to accept the DSO Framework on 3rd March 2011.
    \item In accordance with the DSO Framework, the parent body of the JCR is the College.
    \item Matters on which agreement cannot be reached between the College and the JCR shall be adjudicated by a panel established by agreement of the JCR President and the Head of College for that purpose.
\end{enumerate}
\newpage

\section{Status of Standing Orders}
\begin{enumerate}
    \item The JCR must be operated in accordance with the DSO Framework and its Standing Orders. Should there be a conflict between the Framework and the Standing Orders, the requirements of the Framework will take precedence.
    \item Any documents, policies or decisions of the JCR must be subject to the Standing Orders.
    \item These Permanent Standing Orders may not be suspended.
    \item The Standing Orders must be reviewed at least once every five years.
    \item Where a conflict exists between sections of the Standing Orders, the order of precedence is a follows:
    \begin{enumerate}[a)]
        \item Permanent Standing Orders
        \item Operational Standing Orders
        \item Committees and Job Descriptions
        \item Appendices to the Standing Orders
    \end{enumerate}
    \item Unless explicitly stated otherwise in the Standing Orders, the Appendices to the Standing Orders are advisory only.
    \item The JCR Chair must rule on matters of interpretation.
    \begin{enumerate}
        \item These rulings must be approved by vote of Constitutional Committee.
    \end{enumerate}
\end{enumerate}
\newpage

\section{Amendment of Standing Orders}
\begin{enumerate}
    \item Any proposed changes to the Standing Orders must be submitted to the JCR Constitutional Committee for its consideration.
    \begin{enumerate}
        \item Constitutional Committee can give or withhold its approval or suggest amendments but cannot veto any proposed change.
    \end{enumerate}
    \item Any proposed amendment to the Permanent Standing Orders must be approved by an Officer of the Executive Committee.
    \item Following consideration by Constitutional Committee, a proposed amendment must be presented to a JCR Meeting for debate.
    \item Proposed amendments to the Standing Orders must finally be approved by referendum in accordance with the procedure set out in the Operational Standing Orders.
    \begin{enumerate}
        \item If the ballot fails to be quorate, then its decision may stand if it received at least two-thirds of the votes cast.
    \end{enumerate}
    \item Changes to the Appendices to the Standing Orders that represent a change in a policy/procedure may only be made with approval by a JCR motion in accordance with the procedure set out in the Operational Standing Orders.
    \item Changes to the Appendices to the Standing Orders that do not represent a change in policy/procedure, such as the addition of new Honorary Life Member or newly approved societies, may be made by the JCR President.
    \item No amendment to Standing Orders may result in the invalidation of a JCR decision which was reached in accordance with the Standing Orders in effect at the time the decision was made unless explicitly stated.
\end{enumerate}
\newpage

\section{Purpose of the JCR}
\begin{enumerate}
    \item The purpose of the JCR is:
    \begin{enumerate}[(a)]
        \item to contribute to the education of its members;
        \item to provide opportunities for participation in intellectual, cultural and social activities;
        \item to support the welfare of its members;
        \item to provide, in co-operation with the College, facilities, services and opportunities for recreation (including the maintenance of a Common Room);
        \item to act as a channel of communication between its members and the College and other bodies of the University;
        \item to represent its members in matters relating to the government and welfare of the College;
        \item to provide opportunities for its members to develop leadership, organisational and other skills;
        \item to be accountable and transparent to its members;
        \item to use its resources fairly and effectively for the benefit of its members.
    \end{enumerate}
    \item No member of the JCR may support any activities which could bring into disrepute the JCR, the College or the University.
    \item The JCR is opposed to, and will take steps to combat, all forms of unfair discrimination on the grounds of age, appearance, caring responsibilities, caste, class, educational background or current educational status, gender, health status, marital or family status, nationality, political beliefs, religion, immigration status, race/ethnicity, sexuality, irrelevant criminal conviction, physical or mental ability or trade union activity.
\end{enumerate}
\newpage

\section{Membership}
\begin{enumerate}
    \item The following people are full members of the JCR:
    \begin{enumerate}[(a)]
        \item All undergraduate students of Durham University who are registered as members of Van Mildert College and have not opted out of JCR membership.
        \item Sabbatical officers of the JCR.
        \item Any sabbatical officer of a student organisation recognised by the University who was within the twelve months before their appointment as sabbatical officer a member of the JCR.
        \item Any member of the MCR who has opted-in to the JCR.
    \end{enumerate}
    \item There are also Honorary Life Members appointed by the awarding panel.
    \begin{enumerate}
        \item Honorary Life Members are not permitted to be officers of the JCR.
    \end{enumerate}
    \item All members of the JCR, other than Honorary Life Members and sabbatical officers of the JCR, are required to pay a levy. This levy shall be collected in the first term of that member’s first year of study. Persons who have not paid the levy shall be considered to have opted out of JCR membership.
    \item Any member may opt out of JCR membership at any time by providing a signed statement to that effect to the JCR President.
    \item A person who has opted out of JCR membership may opt in to JCR membership by providing a signed statement to that effect to the JCR President and paying the JCR levy if they have not previously done so.
    \item Only JCR members are entitled to vote on decisions of or to be an officer of the JCR (or any club or society affiliated to it).
\end{enumerate}
\newpage

\section{Business}
\begin{enumerate}
    \item Decisions of the JCR must normally be made at a JCR meeting at which all members of the JCR are entitled to attend.
    \item The JCR Chair must ensure that there are at least two ordinary meeting of the JCR per academic term.
    \item At least five days’ notice must be given for an ordinary meeting of the JCR. 
    \item At least three days’ notice must be given for an extraordinary meeting of the JCR, which may be called by:
    \begin{enumerate}[(a)]
        \item A signed request from JCR members with the signatures of at least 10\% of JCR members.
        \item JCR Executive Committee.
    \end{enumerate}
    \item At least 24 hours’ notice must be given for an emergency meeting, which may be called by the Executive Committee in the case where an immediate decision by the JCR is required.
    \item At least three days’ notice must be given for a referendum or election.
    \item Where “notice” is required to be given for the purposes of subsections 6.3-6.6, this must be given by an email to all JCR members and by a post on the JCR Facebook group. It is the joint responsibility of the Chair and Communications Officer that such notice is given in sufficient time.
    \item Where “notice” is required to be given for the purposes of subsections 6.3-6.5, this must include the agenda for the meeting. If any changes are subsequently made to the agenda prior to the meeting, the Chair should make this known at an appropriate time.
    \item All members are invited to attend meetings of the JCR and all have an equal right to vote and to speak on any matter. The Operational Standing Orders set out the procedures by which business is considered at JCR meetings.
    \item A meeting of the JCR shall be considered quorate if 7\% of the members are present.
    \item In the event of a JCR meeting being inquorate, the meeting may not vote on motions or elections.
    \item Unless otherwise stipulated by the Standing Orders, decisions must be made by simple majority vote.
    \item Any motion, normal or financial, may be referred to the Executive Committee or Treasury Committee respectively for a ruling if a decision is required before the next quorate JCR meeting.
    \begin{enumerate}
        \item If passed, the motion must be debated at the next JCR meeting.
    \end{enumerate}
    \item All policy passed by the JCR must be renewed annually in a JCR meeting and must be incorporated into Appendix I of the Standing Orders.
    \item A referendum of the JCR shall be considered quorate if 10\% of eligible members vote.
    \item In the event of a referendum failing to be quorate, the decision shall be invalid unless a provision in the Standing Orders states otherwise.
    \item The JCR has an Executive Committee which is collectively responsible to the JCR for:
    \begin{enumerate}[(a)]
        \item Administering the JCR during the periods between JCR Meetings. This will include making decisions on behalf of the JCR on routine or non-contentious matters.
        \item Making decisions for which the deadline does not allow consideration at a JCR meeting.
        \item Representing the JCR to the College and wider University community.
        \item Ensuring the proper conduct of JCR Officers and committees.
        \item Ensuring that the JCR is maintained in a sound financial position and that appropriate financial records are being maintained.
        \item Managing the business of the JCR meeting.
    \end{enumerate}
    \item The Executive Committee consists on the following members:
    \begin{enumerate}[(a)]
        \item JCR President
        \item JCR Vice-President (Development)
        \item JCR Vice-President (Welfare)
        \item JCR Treasurer
        \item JCR Communications Officer
        \item JCR Events Officer
        \item JCR Services Manager
        \item JCR Senior DSU Representative
        \item JCR Senior Freshers’ Representative
        \item JCR Sports and Societies Officer
        \item JCR Outreach Officer
        \item JCR Chair
        \item Sabbatical Bar Steward
    \end{enumerate}
    \item The Executive Committee, whenever possible, will be chaired by the JCR Chair
    \begin{enumerate}
        \item When not possible the order of precedence for chairing Executive Committee is as follows:
        \begin{enumerate}[(a)]
            \item The JCR Vice-Chair
            \item The JCR President
            \item The JCR Vice-President (Development)
            \item The JCR Vice-President (Welfare)
            \item The Communications Officer
            \item Any other member of the Executive Committee
        \end{enumerate}
    \end{enumerate}
    \item The Executive Committee must submit reports on its actions to JCR meetings as appropriate.
    \item The JCR may delegate responsibilities to committees and officers as set out in Operational Standing Orders.
    \item Without prejudice to any indemnity to which any Executive Officer may otherwise be entitled, every Executive member or other officer of the student body shall be indemnified, out of the assets of the student body, against any liability incurred by them in defending any proceedings, whether civil or criminal, in which judgment is given in their favour or in which they are acquitted, or in connection with any application in which relief is granted to them by the court from liability for negligence, default, breach of duty, or breach of trust in relation to the affairs of the students.
    \item The JCR must not censor the speech, writing or other works of its members unless they are grossly offensive, misleading disclosing confidential information, defamatory, illegal, or otherwise likely to result in legal action being against the JCR, except where this is a disclosure of information showing that wrongdoing has been committed, or where the speech is commercial in nature.
\end{enumerate}
\newpage

\section{Appointment and Removal of JCR Officers}
\begin{enumerate}
    \item The term "JCR Officers" refers to members of the Executive Committee and any other position explicitly defined as being so in the Standing Orders.
    \item All JCR Officers must be elected by the JCR.
    \item In all elections “Re-open Nominations” (RON) must be included as an option.
    \item Executive Officers (i.e. those officers of the JCR who are members of the Executive Committee) and Non-executive Officers (i.e. those officers of the JCR who are not members of the Executive Committee) must be appointed using Single Transferable Voting in accordance with the Operational Standing Orders.
    \begin{enumerate}
        \item An election shall be considered quorate if 10\% of members have voted.
    \end{enumerate}
    \item Members of JCR Committees shall hold office until the end of the academic year in which they are appointed, or for a period as determined by the Committees and Job Description Standing Order.
    \item Any non-sabbatical Officer of the JCR may stand down from their role by writing to the Executive Committee. The Executive Committee must report this to the next JCR meeting.
    \item Should a JCR Officer be considered to have fallen short of fulfilment of the duties assigned to them, a quorate JCR meeting may agree a formal warning against that officer.
    \item Should a JCR Officer continue to fall short of fulfilment of the duties assigned to them following a formal warning, or if the actions of an officer are considered to be serious misconduct, a motion of no confidence must be debated at a JCR meeting and then put to a referendum.
    \begin{enumerate}
        \item If a motion of no confidence is agreed, the officer shall immediately cease to hold their office.
        \item The Executive Committee may suspend the officer from carrying out their duties until the referendum has been voted on.
    \end{enumerate}
    \item Any vacancy which arises must be filled at the earliest opportunity by an election conducted in the manner normal for that post. The vacancy shall be filled for the remainder of the original term of office.
    \begin{enumerate}
        \item Any officer appointed in this manner shall be eligible to stand for re-election for a full term of office at the normal time should they continue to meet any conditions for election to that role.
    \end{enumerate}
    \item The following special conditions apply to the appointment of sabbatical officers of the JCR:
    \begin{enumerate}
        \item Following their election, their appointment shall only be valid following the signing of a contract of employment with the University.
        \item Sabbatical officers may resign from their post by giving written notice of one term to the Executive Committee and the Director of HR of the University.
        \item If a sabbatical officer is accused of committing a serious offence that falls within the University’s definition of gross misconduct (as set out in the University’s Disciplinary Regulation) the JCR Executive will consider a motion of no confidence.
        \item In the event of a motion of no confidence in a sabbatical officer, a panel will be convened to consider the case for dismissal from office. The membership of the panel shall be agreed by the members of the JCR Executive Committee in consultation with the Director of HR (or their deputy) of the University.
        \item Should a sabbatical officer be dismissed from Office they have the right of appeal. The appeal process will be determined by the JCR Executive in consultation with the HR Director of the University (or their deputy).
        \item Where there is a vacancy for a sabbatical officer, it shall be filled for the remainder of the term of office by a non-sabbatical office elected in the normal manner. An individual appointed in these circumstances shall be eligible to stand for re-election for a full term of office.
    \end{enumerate}
    \item Following informal discussions, if a member considers that there has been misconduct by the JCR President the matter should be discussed with the JCR Chair who must consult with the JCR Executive Committee (other than the President).
    \item Where more than one member of the JCR Executive Committee (excluding the Chair) considers that there may be a case of misconduct by the JCR President, the matter must be discussed with the Head of College.
    \item Following discussion with the Head of College (and where necessary with the Director of HR of the University), if the JCR Executive Committee consider that there has been misconduct, but that misconduct falls short of gross misconduct as defined by the University Regulations, the Executive Committee must implement an appropriate remedy (that complies with the Standing Orders) and report this to the JCR. This may include a vote of no confidence in the President.
    \item Following discussion with the Head of College and the Director of HR, if the JCR Executive Committee consider that there has been gross misconduct as defined by the University Regulations, the JCR must have a vote of no confidence in the President.
    \item If the JCR President ceases to be an employee of the University, then they shall also cease to hold the office of JCR President.
\end{enumerate}
\newpage

\section{JCR Finances}
\begin{enumerate}
    \item The JCR President has overall responsibility for the JCR’s finances and for ensuring that the JCR remains in a sound financial position.
    \item The JCR Treasurer is responsible to the JCR President for the financial transactions of the JCR and must advise the JCR President and the Executive Committee on financial matters.
    \item The JCR Treasurer is responsible for the preparation of the JCR budget and accounts and for liaising with the Colleges Accounts Team and the Head of College as appropriate.
    \item Other JCR officers may be delegated specific financial responsibilities. Their work will be overseen by the President.
    \item The JCR Treasury Committee has the power to intervene in the running of a JCR Committee or Society account if they are making a financial loss or if there is reasonable suspicion of misuse of funds.
    \item The JCR Financial Auditing Committee must have access to all financial documents of the JCR.
\end{enumerate}
\newpage

\section{Resolution of Issues}
\begin{enumerate}
    \item Members of the JCR should attempt to resolve issues by informal discussion before following the formal procedures for resolution of issues. 
    \item Any matter which may represent a criminal act must be reported by the JCR President or JCR Chair to the Head of College who shall liaise with the Registrar.
    \subsection*{Appealing a Decision of the JCR}
    \item If a member of the JCR wishes to appeal against a decision of the JCR (or one of its committees or officers) they may do so by writing to the JCR Chair to request the decision be considered at a JCR meeting.
    \item If, following consideration by the JCR, a member considers that the final decision of the JCR unfairly disadvantages a member or group of members they may raise the matter in writing with the Head of the College, who shall take reasonable steps to resolve the matter. Should this not prove possible, the matter will be referred to the College Council for final resolution.
    \subsection*{Alleged Misconduct by a JCR Member}
    \item If a member considers that there has been misconduct by a member or members of the JCR (while participating in JCR activities), the matter should be discussed with either the JCR President or the JCR Chair. The JCR President or Chair may consult with the Executive Committee if this is considered by the JCR President or Chair to be an appropriate course of action.
    \item Where the JCR President or JCR Chair determines that a breach of the JCR Standing Orders or policy has occurred, they must implement an appropriate remedy (that complies with the Standing Orders), consulting with Constitutional Committee and, where appropriate, Executive Committee.
    \item Where the JCR President or JCR Chair consider that the misconduct may represent a breach of College or University Regulations, the matter must be raised in writing with the Head of College who shall take appropriate action.
\end{enumerate}

\end{document}