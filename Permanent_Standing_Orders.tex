% Set font size to 12; article is default LaTeX doc type and meets our purposes
\documentclass[12pt]{article}
\usepackage{mildert}

\title{Permanent Standing Orders}
\author{Van~Mildert~College Junior~Common~Room}
\newcommand{\thedate}{15th March 2024}
\date{\thedate}

\begin{document}
\begin{titlepage}  % Title page
    \maketitle
    \begin{figure}[h]
    \includegraphics[scale=0.25]{arms}  % Coat of arms
    \centering
    \end{figure}
    \thispagestyle{empty}
\end{titlepage}

\setcounter{page}{2}  % Correct page numbering
\section*{Introduction}
These are the Permanent Standing Orders of Van Mildert College Junior Common Room, approved on \thedate.

This version was revised by the Steering Committee of 2017-18:\\
\hspace*{2cm}Georgina Robinson (JCR Chair) (HLM)\\
\hspace*{2cm}Alastair Hargreaves-McManus\\
\hspace*{2cm}George Singleton\\
\hspace*{2cm}James Smith\\
\hspace*{2cm}Amelia Spencer (JCR Vice-Chair)

Amended in October 2018 by the Executive Committee, Sheehan Quirke (JCR President) and Alastair Hargreaves-McManus.\\
Amended in June 2019 by Alastair Hargreaves-McManus and Sheehan Quirke (JCR President).\\
Amended in June 2020 by the Executive Committee.\\
Amended in November 2021 by Dominic Laurens (JCR Chair).\\
Amended in February 2022 by the Executive Committee and Governance Committee.\\
Amended in May 2022 by the Executive Committee and Governance Committee.\\
Amended in October 2022 by Dominic Laurens (JCR Chair), and the Executive and Governance Committees. \\
Amended in November 2022 by Dominic Laurens (JCR Chair), and the Executive and Governance Committees. \\
Amended in February 2024 by Dylan Jones (JCR Chair).
Amended in March 2024 by Dylan Jones (JCR Chair).
\newpage
\tableofcontents{}
\newpage


\section{Preface}
\begin{enumerate}
    \item The JCR is a students' union formed as a Charitable Incorporated Organisation.
    \item The JCR voted to leave the Durham Student Organisation in February 2019 and was established as an independent charity in August 2020.
    \item The JCR is bound by several agreements signed with Durham University. These have the status of directions of the Board of Trustees.
    \item These include a Service Level Agreement, which binds the JCR to comply with Durham University’s Recognised Common Rooms Code of Practice and their Common Room Complaints Procedure for Members.
    \item As a students’ union, the JCR has several legal obligations under the Education Act 1998.
    \item In these Standing Orders the following terms shall be understood with the following meanings:\\
\begin{tabular}{|p{1.5in}|p{3.8in}|}\hline
    \textbf{Term}           & \textbf{Definition}\\ \hline
    JCR                     & Junior Common Room of Van Mildert College\\ \hline
    CIO						& Van Mildert College Junior Common Room  Charitable Incorporated Organisation\\\hline
    MCR                     & Middle Common Room of Van Mildert College\\\hline
    College                 & Van Mildert College\\ \hline
    DSU                     & Durham Students' Union\\ \hline
    Standing Orders         & Collectively the Permanent Standing Orders, the Operational Standing Orders and Committees and Job Descriptions Standing Orders \\ \hline
    Governing Document      & The constitution of the CIO\\\hline
    Trustees				& Trustees of the CIO\\\hline
    Student Trustees		& Students elected as Trustees\\\hline
    Internal Trustees		& Student and Sabbatical Officer Trustees\\\hline
    External Trustees		& Trustees who are not Internal Trustees\\\hline
    Board of Trustees		& Trustee board of the CIO\\\hline
    Officer                 & Any person who holds an official position in the JCR specified in the Standing Orders\\ \hline
    Elected Officer         & An Officer elected where the electorate was the whole JCR \\ \hline
    Appointed Officer       & An Officer appointed by interview to a position (that is not an ordinary committee member) \\ \hline
    Ordinary committee member & A member of a committee who is not a member ex-officio \\ \hline
    Executive Officer       & An Officer who is a member of the Executive Committee \\ \hline
    Sabbatical Officer		& The President or FACSO\\\hline
    Ballot                  & An election or referendum \\ \hline
    Examination Period      & The Main Examination Period in May/June as defined by the University\\\hline
\end{tabular}
\end{enumerate}
%\newpage

\section{Status of Standing Orders}
\begin{enumerate}
    \item The JCR must be operated in accordance with its Governing Document and its Standing Orders. Should there be a conflict between the Governing Document and the Standing Orders, the Governing Document will take precedence.
    \item The JCR must be operated in accordance with the directions of the Board of Trustees. Should there be a conflict with the Standing Orders, the directions will take precedence.
    \item Any documents, policies or decisions of the JCR must be subject to the Standing Orders.
    \item Where a conflict exists between parts of the Standing Orders, the order of precedence is a follows:
    \begin{enumerate}[(a)]
        \item Permanent Standing Orders
        \item Operational Standing Orders
        \item Committees and Job Descriptions
        \item Appendices to the Standing Orders
    \end{enumerate}
    \item The Committees and Job Descriptions part must only contain descriptions of JCR committees and roles.
    \begin{enumerate}
        \item The part may specify terms of office.
        \item The part may not grant monetary benefits.
    \end{enumerate}
    \item Unless explicitly stated otherwise in the Standing Orders, the Appendices to the Standing Orders are advisory only.
    \item The JCR Chair must rule on matters of interpretation.
    \begin{enumerate}
        \item Significant rulings should be documented and considered by Governance Committee.
    \end{enumerate}
\end{enumerate}
\newpage

\section{Amendment of Standing Orders}
\begin{enumerate}
    \item This section sets out the procedure for amending the Standing Orders.
    \item\label{amendmentConsideration} Any proposed amendment must be considered by the Executive Committee \emph{and} Governance Committee.
    \begin{enumerate}
        \item Consideration may include suggesting amendments or expressing disapproval.
    \end{enumerate}
    \item Any proposed amendment must fulfil one of the following conditions:\label{amendmentApproval}
    \begin{enumerate}[(a)]
        \item Be approved by the Executive Committee and Governance Committee;
        \item Be approved by the Executive Committee without the approval of Governance Committee after having sought the approval of Governance Committee in three different academic terms;
        \item Be proposed by the Board of Trustees; \emph{or}
        \item Be supported by the signatures of at least 25\% of JCR members.
    \end{enumerate}
    \item Any proposed amendment must be approved by the Board of Trustees.\label{amendmentApproval-trustees}
    \item Proposal by a committee is taken to imply its consideration and approval.
    \item\label{amendmentDebate}Following fulfilment of the clauses \ref{amendmentConsideration}, \ref{amendmentApproval} and \ref{amendmentApproval-trustees}, a proposed amendment must be published to members and presented to a JCR Meeting for debate.
    \item Finally, a proposed amendment must be approved by referendum in accordance with the procedure set out in the Operational Standing Orders.
    \begin{enumerate}
        \item If the referendum fails to be quorate, then it may approve the amendment if it was supported by at least two-thirds of the votes cast.
        \item Proposed minor amendments to the Committees and Job Descriptions part only may instead be approved by an ordinary JCR motion at a quorate JCR meeting.
    \end{enumerate}
    \item Changes to the Appendices to the Standing Orders that represent a change in a policy/procedure may only be made with approval by a JCR motion in accordance with the procedure set out in the Operational Standing Orders.
    \begin{enumerate}
        \item Notwithstanding this clause, changes to the list of active and inactive policies within Appendix I may be carried out as outlined in Appendix I.
    \end{enumerate}
    \item Changes to the Appendices to the Standing Orders that do not represent a change in policy/procedure, such as the addition of new Honorary Life Member or newly approved societies, may be made by the JCR President.
    \item No amendment to Standing Orders may result in the invalidation of a JCR decision which was reached in accordance with the Standing Orders in effect at the time the decision was made unless explicitly stated.
\end{enumerate}
\newpage

\section{Purpose of the JCR}
\begin{enumerate}
    \item The purpose of the JCR is:
    \begin{enumerate}[(a)]
        \item to contribute to the education of its members;
        \item to provide opportunities for participation in intellectual, cultural and social activities;
        \item to support the welfare of its members;
        \item to provide, in co-operation with the College, facilities, services and opportunities for recreation (including the maintenance of a Common Room);
        \item to act as a channel of communication between its members and the College and other bodies of the University;
        \item to represent its members in matters relating to the government and welfare of the College;
        \item to provide opportunities for its members to develop leadership, organisational and other skills;
        \item to be accountable and transparent to its members;
        \item to use its resources fairly and effectively for the benefit of its members.
    \end{enumerate}
    \item All JCR Officers are accountable to JCR members.
    \item No member of the JCR may support any activities which could bring into disrepute the JCR, the College or the University.
    \item The JCR is opposed to, and will take steps to combat, all forms of unfair discrimination on the grounds of age, appearance, caring responsibilities, caste, class, educational background or current educational status, gender, health status, marital or family status, nationality, political beliefs, religion, immigration status, race/ethnicity, sexuality, irrelevant criminal conviction, physical or mental ability or trade union activity.
\end{enumerate}
\newpage

\section{Membership}
\begin{enumerate}
    \item The following people are eligible to be full members of the JCR:
    \begin{enumerate}[(a)]
        \item Undergraduate students of the University~of~Durham who are members of Van~Mildert~College.
        \item Postgraduate students who are members of the MCR.
        \item Sabbatical officers of the JCR.
        \item The sabbatical Bar Steward of the College bar, if they were a member of the JCR prior to their starting the role.
        \item Any sabbatical officer of a student organisation recognised by the University who was within a year prior to their taking office a member of the JCR.
    \end{enumerate}
    \item Eligible undergraduate students are assumed to be members of the JCR unless they opt out of membership.
    \item Student members of the JCR are required to pay a levy for membership.
    \begin{enumerate}
        \item Students who fail to pay their levy are considered to have opted out of membership.
    \end{enumerate}
    \item There are also Honorary Life Members appointed by the awarding panel.
    \begin{enumerate}
        \item Honorary Life Members are not permitted to be officers of the JCR.
    \end{enumerate}
    \item Any member may opt out of JCR membership at any time by providing a signed statement to that effect to the JCR President.
    \begin{enumerate}
        \item A member who opts out of membership does not have the right to receive a refund of their levy.
        \item Partial refunds of a member's levy may be given in exceptional circumstances at the discretion of the sabbatical officers.
    \end{enumerate}
    \item A person who has opted out of JCR membership may opt in to JCR membership by providing a signed statement to that effect to the JCR President and paying the JCR levy if they have not previously done so.
    \item A member may have their membership revoked by the Board of Trustees if they are found to have committed serious misconduct.
    \item Only JCR members are entitled to vote on decisions of or to be an officer of the JCR (or any club or society affiliated to it).
\end{enumerate}
\newpage

\section{Business}
\begin{enumerate}
    \item Decisions of the JCR must normally be made at a JCR meeting at which all members of the JCR are entitled to attend.
    \item The JCR Chair must ensure that there are at least two ordinary meeting of the JCR per academic term.
    \item All members are invited to attend meetings of the JCR and all have an equal right to vote and to speak on any matter. The Operational Standing Orders set out the procedures by which business is considered at JCR meetings.
    \item A meeting of the JCR shall be considered quorate if 7\% of the members are present.
    \item If a JCR Meeting is not quorate, it must not make decisions (for example, vote on motions or elections) except in accordance with the following subclauses.
    \begin{enumerate}
        \item Procedural motions take effect immediately and do not require the JCR Meeting to be quorate. A procedural motion must only relate to the conduct of the meeting.
        \item The Chair may allow decisions to be made by an inquorate meeting if they are uncontroversial, they do not permit the use of JCR funds and no member present objects.
        \item No decisions may be made in a JCR Meeting with fewer than 10 members present or fewer than 5 members who not Executive Officers present.
        \item Decisions made by an inquorate JCR Meeting may be voided by the Executive Committee or Governance Committee. This may only happen within 3 days of the decision taking effect.
    \end{enumerate}
    \item Decisions of a JCR Meeting (excluding procedural motions) do not take effect until minutes of the meeting are published.
    \item Unless otherwise stipulated by the Standing Orders, decisions must be made by simple majority vote.
    \item In the event of a referendum failing to be quorate, the decision shall be invalid unless a provision in the Standing Orders states otherwise.
    \subsection{Executive Committee}
    \item The JCR has an Executive Committee which is collectively responsible to the JCR for:
    \begin{enumerate}[(a)]
        \item Administering the JCR during the periods between JCR Meetings. This will include making decisions on behalf of the JCR on routine or non-contentious matters.
        \item Making decisions for which the deadline does not allow consideration at a JCR meeting.
        \item Representing the JCR to the College and wider University community.
        \item Ensuring the proper conduct of JCR Officers and committees.
        \item Ensuring that the JCR is maintained in a sound financial position and that appropriate financial records are being maintained.
        \item Managing the business of the JCR meeting.
    \end{enumerate}
    \item The Executive Committee consists of the following members:
    \begin{enumerate}[(a)]
        \item JCR President
        \item JCR Senior Welfare Officer
        \item JCR Financial and Commercial Services Officer (FACSO)
        \item JCR Communications Officer
        \item JCR Events Officer
        \item JCR Senior Freshers' Representative
        \item JCR Sports and Societies Officer
        \item JCR Outreach and Fundraising Officer
        \item JCR Chair
        \item JCR International Officer
        \item Sabbatical Bar Steward
    \end{enumerate}
    \item The Sabbatical Bar Steward is a non-voting member of the Executive Committee.
    \item The Executive Committee must submit reports on its actions to JCR meetings as appropriate.
    \item Executive Officers are expected to attend all JCR meetings.

    \subsection{Internal Trustees}
    \item The President and FACSO are trustees of the CIO.
    \item The JCR elects 2 student trustees for the CIO.
    \item The student trustees must not be Executive Officers.

    \subsection{Miscellaneous provisions}
    \item The JCR may delegate responsibilities to committees and officers as set out in the Standing Orders.
    \item Without prejudice to any indemnity to which any Executive Officer may otherwise be entitled, every Executive member or other officer of the student body shall be indemnified, out of the assets of the student body, against any liability incurred by them in defending any proceedings, whether civil or criminal, in which judgment is given in their favour or in which they are acquitted, or in connection with any application in which relief is granted to them by the court from liability for negligence, default, breach of duty, or breach of trust in relation to the affairs of the students.
     \item The JCR must not interfere with its members' private freedom of expression.
    \begin{enumerate}
        \item The JCR may take reasonable steps to control expression where appropriate and necessary, such as expression that is:
        \begin{enumerate}[(a)]
            \item being made in an official JCR capacity;
            \item grossly offensive, indecent, obscene, menacing or otherwise illegal;
            \item in violation of University regulations;
            \item infringing copyright, defamatory or otherwise likely to result in legal action being taken against the JCR;
            \item harmfully misleading;
            \item in breach of JCR election or meeting rules in place to ensure the fair running of the JCR;
            \item commercial in nature;
            \item disclosing confidential information.
        \end{enumerate}
    \end{enumerate}
    \item Governance Committee must have sufficient independence from the Executive Committee to effectively carry out its scrutiny roles.
\end{enumerate}
\newpage

\section{Elections and Referenda}
\begin{enumerate}
    \item All JCR ballots must be conducted as secret ballots.
    \item All JCR ballots with more than two options must be conducted using the Single Transferable Vote (STV) system.
    \item The quorum for a ballot is 5\% of the electorate.
    \item At least 24 hours' notice must be given to JCR members at each stage of the ballot process.
    \item Every ballot must have a Senior Returning Officer, Junior Returning Officer and a Scrutineer (``\textbf{election officials}'') who are responsible for ensuring that procedures are carried out correctly.
    \begin{enumerate}
        \item A candidate must not act as an election official for the same election.
        \item No person may hold two or more election official roles for the same ballot.
    \end{enumerate}
    \item The election officials must act impartially and transparently.
    \item The Scrutineer must observe the election to ensure it is conducted fairly and in accordance with the Standing Orders.
    \item The voting period for a ballot must be not shorter than 24 hours.
    \item At the end of the voting period, the Returning Officers must announce the results of the ballot within 12 hours of the close of voting.
    \item A person must not cause a change to the results of a ballot by any action other than the casting of a single lawful vote.
    \begin{enumerate}
        \item Consequently:
        \begin{enumerate}[(a)]
            \item A member must not vote more than once in the same ballot.
            \item The results of a ballot must not be altered by any JCR Officer or Committee.
        \end{enumerate}
    \end{enumerate}
    \item A member must not reveal how another member has voted.
    \item To maintain the secrecy of the ballot, a member must not prove to any person how they have voted.
    \item Members must not engage in unfair election practices.
    \begin{enumerate}
        \item Practices considered unfair include:
        \begin{enumerate}[(a)]
            \item Unduly influencing a member to vote or not vote in a certain way;
            \item Making false statements about a candidate;
            \item Treating or bribery.
        \end{enumerate}
    \end{enumerate}
    \item If an election official has reasonable suspicion that the Standing Orders have been contravened regarding a ballot, then they must refer the matter to Governance Committee.
    \item A ballot may be declared invalid if either the Returning Officers or Governance Committee conclude that the Standing Orders have been contravened regarding a ballot and that it is in the interests of democracy to rerun the ballot.
    \item If a ballot is declared invalid, then—
    \begin{enumerate}[(a)]
        \item The result is void.
        \item Governance Committee must decide whether the whole ballot process must be restarted or whether only the voting itself must be rerun.
        \item The Returning Officers must act in accordance with the decision made by Governance Committee within a reasonable timeframe.
    \end{enumerate}
    \item Executive Officers must act impartially towards candidates during elections.
    \begin{enumerate}
        \item They must not offer support or assistance to any candidate without offering the same to all other candidates in the same election.
    \end{enumerate}
    \item In all elections ``Re-open Nominations'' (RON) must be included as an option.
    \item At the end of each academic year, the most prevalent Senior Returning Officer of ballots in that year must submit a report to the Board of Trustees as to their confidence in the fair and proper conduct of ballots.
\end{enumerate}
\newpage

\section{Appointment and Removal of JCR Officers}
\begin{enumerate}
    \item All officers must be appointed in a fair and transparent process.
    \item All Executive Officers must be elected with the exception of the Sabbatical Bar Steward.
    \item A member can only hold one Executive Officer position at a time but can be re-elected to the same position or elected to a different position for a term of office after the end of their current term.
    \item Members of JCR Committees hold office until the end of the academic year in which they are appointed, unless otherwise stated in the Standing Orders.
    \item Any non-sabbatical Officer of the JCR may stand down from their role by writing to the President.
    \begin{enumerate}
        \item The President must report this to the JCR as soon as appropriate.
    \end{enumerate}
    \item Should an Officer be considered to have fallen short of fulfilment of the duties assigned to them, the JCR may pass a formal warning motion against that Officer.
    \begin{enumerate}
        \item Elected Officers may only receive a formal warning via a motion at a JCR meeting.
        \item Appointed Officers may only receive a formal warning at the behest of Governance Committee.
        \item The JCR must be notified of any formal warnings as soon as appropriate.
    \end{enumerate}
    \item Should an Officer continue to fall short of fulfilment of the duties assigned to them following a formal warning, or if the actions of an Officer are considered to be serious misconduct, their removal from office must be considered.
    \begin{enumerate}
        \item Elected Officers may only be removed from office by a motion of no confidence at a JCR meeting, followed by a referendum.
        \item Appointed Officers may only be removed from office at the behest of both the Executive Committee and Governance Committee.
        \item An Officer may be suspended from their duties, at the behest of the Executive Committee, while their removal is being considered.
        \item An Officer will immediately cease to hold office as soon as their removal is agreed.
        \item The JCR must be notified of the removal of an Officer as soon as appropriate.
    \end{enumerate}
    \item Any vacancy which arises must be filled at the earliest opportunity in the manner normal for that post.
    \item The following special conditions apply to Sabbatical Officers of the JCR:
    \begin{enumerate}
        \item Candidates for Sabbatical Officer positions must be in the final year of their degree or be an incumbent Sabbatical Officer.
        \item A member may only hold Sabbatical Office for up to two years in total, in accordance with the Education Act 1994 section 22.
        \item Sabbatical Officers may resign from their post in accordance with their contract of employment.
        \item If a Sabbatical Officer is accused of misconduct, this must be reported to the Board of Trustees.
        \item Where there is an extraordinary vacancy for a Sabbatical Officer position, a student may be elected to the position for the remainder of the term of office.
    \end{enumerate}
    \item External Trustees may also be subject to motions of no confidence.
    \item Trustees and Sabbatical Officers have right of appeal against removal in accordance with the Governing Document.
    \item The following special conditions apply to the Sabbatical Bar Steward:
    \begin{enumerate}
        \item A formal warning motion or motion of no confidence against the Sabbatical Bar Steward must only be with regard to their duties as a JCR Executive Officer and not their employment with the University.
        \item If a motion of no confidence against the Sabbatical Bar Steward is agreed, then they cease to be an Executive Officer.
    \end{enumerate}
\end{enumerate}
\newpage

\section{JCR Finances}
\begin{enumerate}
    \item The trustees are responsible to the JCR for ensuring that the JCR remains in a sound financial position.
    \item The FACSO is responsible for the financial transactions of the JCR and must advise the President and the Executive Committee on financial matters.
    \item The FACSO is responsible for the preparation of the JCR budget and accounts and for liaising with the University on these as required.
    \item Other officers or members of Finance Committee may be delegated specific financial responsibilities overseen by the FACSO.
    \item Finance Committee has the power to intervene in the running of a JCR Committee or Society account if they are making a financial loss or if there is reasonable suspicion of misuse of funds.
\end{enumerate}
\newpage

\section{Resolution of Issues}
\begin{enumerate}
    \item Members of the JCR should attempt to resolve issues by informal discussion.
    \item The complaints process of the JCR is Durham University’s Common Room Complaints Procedure for Members.    
    \subsection{Appealing a decision of the JCR}
    \item If a member of the JCR wishes to appeal against a decision of the JCR (or one of its committees or officers) they may do so by writing to the Chair to request the decision be reviewed by the Executive Committee.
    \item If a member is unhappy with the outcome of a review, they may appeal to the Board of Trustees.
    \subsection{Alleged misconduct by a JCR member}
    \item If a JCR Officer reasonably believes that a member has committed misconduct while participating in JCR activities or otherwise of interest to the JCR, then they must report this to the President or the Chair.
    \item The President and Chair must record and respond to reports of misconduct appropriately.
    \item Any matter which may represent a criminal act must be reported by the President or Chair to the Head of College and the Board of Trustees.
    \item Members must not unreasonably obstruct investigations of alleged misconduct.
    \item Where the JCR President or JCR Chair determines that a breach of the JCR Standing Orders or policy has occurred, they must implement an appropriate remedy (that complies with the Standing Orders), consulting with Governance Committee and, where appropriate, Executive Committee and/or Board of Trustees.
\end{enumerate}
\end{document}