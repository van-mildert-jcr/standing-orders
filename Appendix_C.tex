\documentclass[12pt]{article}
\usepackage{mildert-common}
\usepackage{enumerate}

\title{Appendix C: Election Candidate Guidance}
\author{Van~Mildert~College Junior~Common~Room}
\date{26th June 2020}

\begin{document}
    \begin{titlepage}  % Title page
        \maketitle
        \begin{figure}[h]
            \includegraphics[scale=0.25]{arms}  % Coat of arms
            \centering
        \end{figure}
        \thispagestyle{empty}
    \end{titlepage}
    \setcounter{page}{2}  % Correct page numbering

    This appendix is advisory only; the Standing Orders provide the authoritative election rules.

    \section{Standing}
    Before a member makes the decision to stand, they should speak to the current holder(s) of the position. They should also speak to the Chair or President about the election procedure.

    Executive Officers must no show bias to any candidate.


    \section{Nominations}
    Before the close of the nominations period, a candidate must submit their nomination including the names of a proposer and a seconder, and a manifesto to the returning officer (normally the Chair).

    The proposer and seconder must be JCR members and must not be or have been Executive Officers.

    This manifesto must be one side of A4. They may also submit a plain text accessible version.

    Nominations for a sabbatical role must also include a policy page (again one side of A4).

    Candidates for a sabbatical role must be finalists.

    Candidates for President or Senior Welfare Officer must also meet with College Officers.

    \section{Hustings}
    Candidates must take part in hustings at a JCR meeting. This can be done by video call if necessary.

    Candidates for President must prepare a campaign video. This video must be approved by Constitutional Committee prior to the hustings and will be published by the JCR after the hustings.

    \begin{enumerate}
        \item Each proposer must give a one minute speech followed by the candidate giving a two minute speech.
        \item Each candidate must sing a song, perform a poem \emph{or} tell a joke. This may be done with their proposer and/or seconder.
        \item Presidential candidates' videos are played.
        \item Each candidate may each ask one question.
        \item The incumbent may ask questions.
        \item Questions will be taken from the floor.
        \item Candidates may then ask further questions at the discretion of the Chair.
        \item Presidential candidates may make some final concise comments.
        \item If there is a statement for RON, this is read by the Chair or the permitted member.
    \end{enumerate}

    All candidates may respond to questions addressed at an individual candidate.

    \section{Canvassing}
    Candidates may canvass on up to three days totalling 4.5 hours, monitored by Constitutional Committee.

    Campaigning on behalf of candidates or potential candidates for election is forbidden.

    \section{Manifestos}
    The Senior Returning Officer will print manifestos for each candidate and publish them on the JCR Facebook group. Candidates must only place their manifestos in the bar toilets and the communal residential block toilets excluding Deerness (note: Tunstall does not have any communal toilets).

    \section{Voting}
    Voting is conducted using the online voting system provided by the University.
    The voting period is set by the Senior Returning Officer and will be between 24 and 168 hours.

    Candidates should not unreasonably hassle members to vote and must not loiter by the voting station.

    A member must not reveal how another member has voted.

    \section{Results}
    After the close of voting, the Returning Officers will count the votes and announce the results on the stairs outside the bar/JCR.

    Successful Presidential candidates traditionally perform a \emph{Kazoo}.

\end{document}